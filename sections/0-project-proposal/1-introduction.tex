\section{Project description}
\label{sec:proj_description}

Gentrification is a process that occurs when wealthy individuals move into poorer urban areas. Neighbourhoods adapt over time to adjust to the needs and goals of their residents. This jump in wealth can cause neighbourhood culture to change and as such to price out longtime residents. However, gentrification also brings investment, potentially improving neighbourhoods that have long been neglected. 

So which aesthetics do we associate with gentrification? Streetswipe1 was developed as a project to identify what we perceive as being gentrified. By swiping pictures of storefront users are invited to decide on what they consider to be signs of gentrification. However, this only gives us information on what storefronts are considered gentrified.

Many methods for predicting labels from urban imagery have been proposed in recent years, but these utilise uninterpretable deep methods2,3,4. By utilising the Streetswipe dataset together with more interpretable computer vision methods we hope to identify human intuitive patterns. What matters for our perception? Shape and colour? The type of store? As such we hope to answer the question: What visual elements are indicators of perceived gentrification?