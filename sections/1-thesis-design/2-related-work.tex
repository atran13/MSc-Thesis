\section{Related Work}
\label{sec:related_work}

% Current research is scarce when is comes to understanding the textual attributes of gentrification outside of the US, much less studies that leverage the potential of machine learning and streetview imagery. The following sections clarify this gap in research.

% \subsection{Gentrification}

% In examining which factors lead to gentrification based on the existing body of research, Rigolon and Németh \cite{rigolon2019} devised a socioecological model in which people, place, and policy are identified as variables that together shape neighborhood change. The \textit{people} layer includes the neighborhood's demographic and socioeconomic characteristics, such as race, income, and the strength of the community culture. For instance, a high proportion of people of color can limit the investment the area receives from developers, as well as how fast it is "discovered" by gentrifiers, due to stigma and perceptions regarding safety. The \textit{place} layer comprises the neighborhood characteristics, such as its proximity from downtown, presence of old but appealing buildings, presence of green space, accessibility by public transit, schools, offices - all of which gives the neighborhood more redevelopment potential and higher likelihood of gentrification. Lastly, the \textit{policy} layer refers to actions by the authorities and other local interventions to prevent gentrification and displacement. While efforts such as providing and protecting housing mitigate gentrification, investing in public infrastructure in low-income neighborhoods leads to the opposite effect.

% Döring and Ulbricht \cite{döring_ulbricht_2018} define gentrification as having four dimensions: functional, architectural, social, and symbolic. 
% \begin{itemize}
%     \item The functional aspect refers to the establishment of services, businesses, and cultural institutions, often with a recognized name and better quality than those existing in the neighborhood. An example is a supermarket chain, as opposed to an ethnic grocery store.
%     \item The architectural aspect concerns upgrades made to the built environment, and the changing aesthetics that comes along with them. Such upgrades is done to residential as well as social infrastructures, such as public buildings, sidewalks, parks, etc. 
%     \item The symbolic aspect refers to the new image of the gentrifying neighborhood created by the new residents as well as other stakeholders, such as investors and the media, and communicated to the wider public.
%     \item The social aspect concerns the displacement of existing residents and replacement by those with more socioeconomic capital. Economically marginalized by rising costs, and socially marginalized by a changing neighborhood culture, long-established groups of residents find themselves no longer belonging, and the demographic of the neighborhood thus gradually transforms.
% \end{itemize}

% As has been noted by Feiereisen and Sassin \cite{feiereisen2021}, the functional, architectural, and symbolic aspects can be seen as constituting the visual indicators of gentrification. While these three aspects take shape in multiple elements and characteristics of the built environment, textual signage is the one element where they converge. A storefront signage would ideally communicate the functionality of the place, while having a certain style of design that reflects and fits in with the overall architectural design of the place, and altogether convey a certain symbolic image, one that gives passerby an impression of what the place stands for, its atmosphere and aesthetics. 

% \begin{displayquote}
%     "Shop signs are public texts that communicate what stores sell, who is perceived to be on the street and what their commercial desires are thought to be. [...] Similar to spoken utterances and all written texts, signs are designed for particular audiences [...]. Well-crafted stories are place-making tools inasmuch as they maintain and reproduce prevailing cultural standards and values" \cite{trinch_signsays_2017}. 
% \end{displayquote}

% Analyzing storefront signage can thus reveal a lot about the visual as well as semantic pattern of perceived gentrification, and this is where the interest of the current research lies.

% \subsubsection{Gentrification in Amsterdam}

\subsection{Storefront signage and gentrification}

As aforementioned, the current body of research about storefront signage aesthetics in relation to gentrification exists largely in the context of the US. To clarify, these are studies that took a holistic approach and considered multiple elements, namely font type (or typeface), colors, text density, language, and meaning. For a sign to convey functional, architectural, and symbolic features of gentrification, it is not up to one single attribute of its appearance, but rather these multiple elements coming into play. Functions would not be expressed effectively through text font and color, but more through semantic meaning, while the opposite can be said about architectural design. And as neighborhood residents see a store sign, they do not just notice one thing, but rather take in its color, font, and meaning as a whole, and in turn form their perception of the place. Therefore, it is deemed necessary that the current analysis take into account multiple attributes of storefront signage, in order to capture as completely as possible the visual cues of gentrification, as experienced by any observer.

Findings from Brooklyn, New York\cite{trinch_signsays_2017, snajdr_oldschool_2018, snajdr_preserve_2022} show that non-gentrified signage - or as the authors call them: \textit{old-school} signage - typically is text-dense and has larger typeface; names that refer to the location, the owner's name, the type of business, products or services; languages other than English; complementary symbols or images; reference to religion, ethnicity, country of origin, and race. On the contrary, gentrified signage - or \textit{distinction-making} signage - has shorter texts, written in smaller font sizes, lower case letters; more cryptic or ambiguous names, sometimes polysemic and with word-plays; languages other than English that shows sophistication and worldliness. In parallel, a study from Cincinnati, Ohio \cite{rahman2020} also found that signage form and character became more homogeneous as neighborhoods became gentrified.

Next to this, other studies have also analyzed storefront signage but with their focus on one individual attribute. A popular topic within gentrification in this regard is the linguistic landscape of a city or neighborhood, for which storefront signage serves as the most abundant proof. Examples can be found in Seoul, South Korea \cite{hong2020}: in a small ethnic neighborhood called Garibong-dong, where the majority of the Chinese population in Seoul reside, Korean signage has been gradually replaced by Chinese signage throughout the years, mirroring both the demographic composition and the social and economic standing of these groups. Similarly, more English signage has appeared in a commercial neighborhood in Phnom Penh, Cambodia, seemingly displacing French as a second language - a trend observed alongside globalization, gentrification, a generational change in attitudes, and education policies \cite{kasanga2012}. Besides languages in signage, typeface has also been found to be highly correlated with average household income of the corresponding neighborhoods in London \cite{ma2019}.

All in all, it is worthwhile to expand knowledge on the topic of storefont signage and gentrification to other regions, not only because there are clear differences between the aesthetics of gentrified and non-gentrified signage, but also because these patterns of change could be different among countries and cultures. The final part of this section discusses the resources at hand for this task, namely streetview imagery (in conjunction with the Streetswipe dataset of Amsterdam), and machine learning in studying gentrification.

\subsection{Machine learning and Streetview imagery}

There exist multiple gentrification research using machine learning techniques: predictive models have been developed for gentrification in Sydney \cite{thackway2021} and London \cite{reades2019}, based on historic socio-economic transitions. Coupled with the increasing availability of street-level image data - or streetview imagery, this has enabled researchers of urban studies to study visual attributes and patterns of change overtime in cities on a significantly larger scale. Besides the previously mentioned studies on linguistic landscape in Seoul \cite{hong2020} and typeface and neighborhood income in London \cite{ma2019}, which utilized streetview imagery and machine learning, other work has been done to visually measure gentrification via documenting changes \cite{ravuri2022}, detect \cite{huang2022} and deep-map gentrification to reveal areas unknowingly becoming gentrified \cite{ilic2019}. Systematic literature reviews on streetview imagery and computer vision for urban analytics \cite{biljecki_2021, zhanga2023} have noted the applicability and increasing importance of the approach in generating insights and decision-making, while pointing future research to further analyzing written languages in images, as well as between-place inference (applying a machine learning model trained with one area to another area). By studying storefront signage of gentrified/non-gentrified labelled data and applying the learnt attributes to a larger set of Amsterdam data, this study adds to this gap in research. Additionally, computer vision research done with storefront signage so far has only been to classify points of interest \cite{noorian2020, bakaev2019}, whereas the current study utilized these resources to understand the visual attributes of signage by analyzing typeface, color, and semantic.

% - SVI and crowdsourced data for urban visual intelligence / place perception, e.g Place Pulse (Boston, New York, Linz, and Salzburg), Place Pulse 2.0 (a more globally diversed version)
% --> More diversed opinions/view points; also more consistent and reliable
