\section{Conclusion}
\label{sec:conclusion}
% Answer each research question and address how the limitations of the study qualify the conclusion.
% Write your conclusion here. Be sure that the relation between the research gap and your contribution is clear. Be honest about how limitations in the study qualify the answer on the research question.

Answering RQs:

1. CRAFT was able to detect signage with reasonable accuracy and completeness, though noise remains.

2. A fine-tuned ResNet50 is able to learn signage features and classify them with 70\% accuracy.

3. There is a gap between the model's performance on StreetSwipe and on extended data of other gentrified/non-gentrified neighborhoods. This suggests that "statistically" gentrified does not necessarily mean visually gentrified. This points to question about the neighborhood's demographic makeup. It could be the case that displacement did not happen to old residents, or it did happen but old business owners were able to cater to new residents and remained in the gentrified neighborhoods.

4. Characteristics of signage as classified by the model are in-line with findings of previous research: gentrified signage looks more homogenized, and are more likely to be in English, while non-gentrified signs look more decorated with serif fonts and more colors, and are largely in Dutch, with more appearances of other languages.

5. Misclassified signage follow the styles of the class they were classified to (i.e. non-gentrified signage could have minimalist use of fonts and colors, with made them appear gentrified). 

6. Signage of the extended dataset follow the style of the class they were classified to (they look like the correctly classified instances from StreetSwipe). Suggests model is generalizable to detect gentrification through signage, based on visual perception.