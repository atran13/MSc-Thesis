\section{Introduction}
\label{sec:introduction} 

Within urban studies, gentrification is a phenomenon widely discussed. First coined by British sociologist Ruth Glass in 1964 in her work about the inner city of London \cite{Glass1964}, the term refers to a neighborhood changing as a result of wealthier residents moving in, gradually displacing existing residents as local housing and service prices increase, and cultures homogenized or replaced. Gentrification thus involves an economic and demographic shift, as well as changes in the aesthetics of the built environment. For its negative effects on marginalized communities, it is worthwhile to understand and detect gentrification. This study took a focus on the visual indicators of gentrification, as visual elements are arguably the most telling factor of a neighborhood’s cultural identity, demographic and economic characteristics.

Gentrification is a multi-dimensional and multi-step process. Döring and Ulbricht \cite{döring_ulbricht_2018} defined gentrification as having 4 aspects: functional (establishment of businesses and cultural institutions), architectural (upgrade to the built environment), social (marginalization, displacement and replacement of existing residents), and symbolic (communication of a new image of the neighborhood to the wider public). As has been noted by Feiereisen and Sassin \cite{feiereisen_sounding_2021}, the functional, architectural, and symbolic aspects can be seen as constituting the visual indicators of gentrification. While these aspects take shape in multiple characteristics of the built environment, storefront signage (hereafter: signage) is a rich communication medium that embodies all three. It is through signage that businesses directly establish their presence, communicate their commercial purposes and values, and distinguish themselves via curated aesthetics \cite{rahman_signage_2020}. Furthermore, businesses understand the socio-cultural values and identity of the neighborhood, and thus design their appearance to best attract and serve this audience:

\renewcommand{\quote}{\list{}{\rightmargin=\leftmargin\topsep=7pt}\item\relax}
\begin{quote}
    "Shop signs are public texts that communicate what stores sell, who is perceived to be on the street and what their commercial desires are thought to be. [...] Similar to spoken utterances and all written texts, signs are designed for particular audiences [...]. Well-crafted stories are place-making tools inasmuch as they maintain and reproduce prevailing cultural standards and values." \cite{trinch_signsays_2017}
\end{quote}

It can thus be expected that once there is a change in the neighborhood - demographically and economically (i.e. gentrification) - signage would mirror this change. Analyzing signage can help understand gentrification, and here lies the interest of this study.

Existing research into the visual indicators of gentrification often analyzes improvements in the neighborhood's physical appearance, and changes in architecture style \cite{huang_detecting_2022, ravuri_gsv_2022, naik_computer_2017, ilic_deepmap_2019}. Comparisons have been drawn in terms of old versus new features, openness of the properties (e.g. boarded up windows, fences), greenery, colors,...; but not as much attention has been paid to signage as a standalone feature. Most research that has been done in this regards are in the context of the US, in which clear distinctions were noted between gentrified and non-gentrified signage \cite{trinch_signsays_2017, snajdr_oldschool_2018, snajdr_preserve_2022, rahman_signage_2020}. The vast majority of these studies on signage employ qualitative methods, whereby the researchers conduct observational data collection (i.e. manually photographing facades) and summarize what is present in their samples. While their findings provide invaluable insights, their methodologies are undoubtedly labor-intensive. Furthermore, as has been pointed out by Reades et al. \cite{reades_understanding_2019} and Barton \cite{barton_exploration_2016}, such selection of neighborhoods and facades per neighborhood often comes with limitations in terms of generalizability on a city-wide scale. Conclusions were made about the most differentiating characteristics, but to which extent are these characteristics present in the neighborhood, and in the city? Can it really be assumed that all signage from a gentrified neighborhood look the same? If not - i.e. if there exist non-gentrified storefronts in a gentrified neighborhood - what can be said about the actual state of gentrification, such as in terms of the neighborhood demographic makeup? These are some nuances that existing research has not discussed, presumably due to their selection of data and limiting methodologies.

On the other hand, the availability of street view data and machine learning techniques has led to developments in Urban Visual Intelligence \cite{zhanga_urban_2023}, whereby cities' built environments are understood in conjunction with socio-economic circumstances and residents' activities on a large scale. Besides predictive modelling for gentrification \cite{thackway_build_2021, reades_understanding_2019}, work has been done to visually measure gentrification via documenting changes \cite{ravuri_gsv_2022}, detect \cite{huang_detecting_2022} and deep-map to reveal gentrifying areas \cite{ilic_deepmap_2019}. A small number of machine learning research took a focus on signage, but in terms of linguistic landscape \cite{hong_linguistic_2020, thung_detecting_2022}, typeface (font type) \cite{ma_typef_2019}, or to classify points of interest \cite{noorian_detect_2020, bakaev_stsem_2019}, instead of to understand the aesthetics of signage. Systematic literature reviews on Urban Visual Intelligence \cite{biljecki_street_2021, zhanga_urban_2023} have noted its importance in generating insights and decision-making, while pointing future research to analyze written languages in images, as well as between-place inference (applying a machine learning model trained with one area to another). This study positions itself in this research gap, where it aims to understand signage aesthetics as a mirror of gentrification in Amsterdam, while utilizing computer vision and street view imagery to overcome limitations of existing urban ethnographic research.

The dataset at the center of this research was from the StreetSwipe project. Using crowdsourcing, StreetSwipe \cite{streetswipe} lets people decide whether facades in Amsterdam appear gentrified. With this data, the study drew conclusions based on the subjective and common perception of a diverse group of people - arguably a necessity when it comes to a multi-faceted phenomenon such as gentrification. Moreover, with the images taken from all over the city, the results were not constrained per neighborhood. Understanding the visual state of gentrification would help detect potentially gentrifying areas. As a pilot case, the main goal of the study was to see to which extent a computer vision model can learn the visual perception and, correspondingly, classify signage as perceptually gentrified or non-gentrified. Subsequently, with data from areas of the city that was not covered in StreetSwipe, the model was tested to quantify the extent to which the perception hold against the actual state of the neighborhoods (i.e. given all signage from a gentrified neighborhood (as per census data), how much of the signage would be visually perceived as gentrified?). Lastly, via inspecting the model's classifications, insights were provided into the characteristics of gentrified and non-gentrified signage in Amsterdam, as well as cases where the distinction was less clear.

The research question of this study was stated as follow: 

\noindent\textit{To which extent can a computer vision model learn the differences between perceptually gentrified and non-gentrified storefront signage? How do the learned characteristics generalize to other neighborhoods of the city? Lastly, what are the characteristics that distinguish between gentrified and non-gentrified signage, and which characteristics make the distinction less clear?}

\begin{enumerate}
    \item Sub-RQ 1: To which extent can the scene text detection model CRAFT identify signage from street view images?
    \item Sub-RQ 2: To which extent can fine-tuned ResNets correctly classify gentrified and non-gentrified StreetSwipe signage?
    \item Sub-RQ 3: How does the model trained on visual perception (StreetSwipe) perform on a test set of data from other neighborhoods, labeled per census data (the extended dataset)?
    \item Sub-RQ 4: What are the characteristics of correctly classified StreetSwipe signage per class with classification probability of 80\% and above?
    \item Sub-RQ 5: What are the characteristics of misclassified Street-Swipe signage with high ($ \geq 80\% $) and low (50-70\%) classification probability?
    \item Sub-RQ 6: What are the characteristics of signage in the extended dataset with classification probability of 80\% and above (irregardless of the ground truth)? Compared to the corresponding facades, to which extent are the classifications plausible?
\end{enumerate}

The paper continues with a description of related work, the study's methodology, results, discussion, and conclusion.