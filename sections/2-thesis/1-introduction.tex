\section{Introduction}
\label{sec:introduction}
% Mention scientific context/field, problem statement, research gap and candidate (sub) research question(s). 

Within urban studies, gentrification is a phenomenon widely discussed. First coined by British sociologist Ruth Glass in 1964 in her work about the inner city of London \cite{Glass1964}, the term refers to a neighborhood changing as a result of wealthier residents moving in, gradually displacing existing residents as local housing and service prices increase, and cultures homogenized or replaced. Gentrification thus involves an economic and demographic shift, as well as changes in the aesthetics of the built environment. For its negative effects on marginalized communities, it is worthwhile to understand and detect gentrification. This study takes a focus on the visual indicators of gentrification, as visual elements are arguably the most telling factor of a neighborhood’s cultural identity, demographic and economic characteristics.

Gentrification is a multi-dimensional and multi-step process. Döring and Ulbricht \cite{döring_ulbricht_2018} define gentrification as having 4 aspects: functional (establishment of businesses and cultural institutions), architectural (upgrade to the built environment), social (marginalization, displacement and replacement of existing residents), and symbolic (communication of a new image of the neighborhood to the wider public). As has been noted by Feiereisen and Sassin \cite{feiereisen_sounding_2021}, the functional, architectural, and symbolic aspects can be seen as constituting the visual indicators of gentrification. While these three aspects take shape in multiple characteristics of the built environment, on-premise signage, or storefront signage (hereafter: signage) is a rich communication medium that embodies all three aspects. It is through signage that businesses directly establish their presence, communicate their commercial purposes and values, and distinguish themselves via their curated aesthetics \cite{rahman_signage_2020}. Furthermore, businesses understand the socio-cultural values and identity of the neighborhood, and thus design their appearance to best attract and serve this audience:

\begin{displayquote}
    "Shop signs are public texts that communicate what stores sell, who is perceived to be on the street and what their commercial desires are thought to be. [...] Similar to spoken utterances and all written texts, signs are designed for particular audiences [...]. Well-crafted stories are place-making tools inasmuch as they maintain and reproduce prevailing cultural standards and values" \cite{trinch_signsays_2017}. 
\end{displayquote}

It can therefore be expected that once there is a change in the neighborhood - demographically and economically (i.e. gentrification) - signage would act as a mirror for this change. Analyzing signage can help understand gentrification, and this is where the interest of the current study lies.

Existing research into the visual indicators of gentrification often analyzes elements such as improvements in the neighborhood's physical appearance and changes in architecture style \cite{huang_detecting_2022, ravuri_gsv_2022, naik_computer_2017, ilic_deepmap_2019}. Comparisons have been drawn between gentrified and non-gentrified facades in terms of old versus new features, openness of the properties (e.g. boarded up windows, fences), greenery, colors,...; but not as much attention has been paid to signage as a standalone feature. Most research that has been done in this regards are in the context of the US, in which clear distinctions were made between gentrified and non-gentrified storefront signage \cite{trinch_signsays_2017, snajdr_oldschool_2018, snajdr_preserve_2022, rahman_signage_2020}. The vast majority of these studies on signage employ qualitative methods, whereby the researchers conduct observational data collection (i.e. manually photographing facades) and summarize what is present in their samples. While their findings provide invaluable insight, their methodologies are undoubtedly labor intensive. Furthermore, as has been pointed out by Reades et al. \cite{reades_understanding_2019} and Barton \cite{barton_exploration_2016}, such selection of neighborhoods and facades per neighborhood often comes with limitations in terms of generalizability on a city-wide scale.

Concurrently, the increasing availability of street view data and machine learning techniques has led to developments in Urban Visual Intelligence \cite{zhanga_urban_2023}, whereby cities' built environments are understood in conjunction with socio-economic circumstances and residents' activities on a large scale. Besides predictive modelling for gentrification based on historic socio-economic transitions \cite{thackway_build_2021, reades_understanding_2019}, work has been done to visually measure gentrification via documenting changes \cite{ravuri_gsv_2022}, detect \cite{huang_detecting_2022} and deep-map gentrification to reveal areas unknowingly becoming gentrified \cite{ilic_deepmap_2019}. A small number of machine learning research took a focus on signage, but in terms of linguistic landscape \cite{hong_linguistic_2020, thung_detecting_2022}, typeface (font type) \cite{ma_typef_2019}, or to classify points of interest \cite{noorian_detect_2020, bakaev_stsem_2019}, instead of to understand the overall aesthetic of signage.

Systematic literature reviews on urban analytics \cite{biljecki_street_2021, zhanga_urban_2023} have noted the applicability and increasing importance of the approach in generating insights and decision-making, while pointing future research to further analyzing written languages in images, as well as between-place inference (applying a machine learning model trained with one area to another area). The current study positions itself in this research gap, where it aims to expand the knowledge of signage aesthetics as a mirror of gentrification in Amsterdam, while utilizing computer vision and street view imagery to overcome limitations of existing urban ethnographic and socio-economic studies.

The dataset at the center of this research is from the StreetSwipe project. Using crowdsourcing, StreetSwipe \cite{streetswipe} lets people decide whether each facade in Amsterdam appears gentrified. By using this data, this study draws conclusions based on the subjective and common perception of a diverse group of people - arguably a necessity when it comes to understanding a nuanced and multi-faceted phenomenon such as gentrification. Moreover, with the images being sampled from locations all over the city, the results are not constrained per neighborhoods. Understanding what people see as gentrified not only means understanding the current state of gentrification, but also would help detect potential areas undergoing gentrification. As a pilot case, the main goal of the study is to see to which extent a computer vision model can learn the visual perception, and correspondingly classify signage as perceptually gentrified or non-gentrified. Subsequently, with added data from more areas of the city that has not been covered in StreetSwipe, the model was tested to quantify the extent to which this perception hold against the actual state of the neighborhoods (i.e. given all signage from a gentrified neighborhood (as per census data), how many of the signage would be visually perceived as gentrified?). Lastly, via inspecting the prediction output, insights are provided into the most typical characteristics of gentrified and non-gentrified signage, as well as cases where the distinction is less clear.

The research question of this study is stated as follow: 

\noindent\textit{To which extent can we learn the visual differences between perceptually gentrified and non-gentrified storefront signage using a computer vision model? How does this visual perception generalize given data from other neighborhoods of the city? Lastly, what are the defining characteristics that distinguish gentrified from non-gentrified signage, and which characteristics make the distinction less clear?}

To answer this research question, this study clarifies the following sub-questions: 

\begin{enumerate}
    \item Sub-RQ 1: To which extent can the scene-text detection model CRAFT identify storefront signage from street view images?
    \item Sub-RQ 2: To which extent can fine-tuned ResNets correctly identify gentrified and non-gentrified signage?
    \item Sub-RQ 3: How does the model trained on visual perception perform on a test set of data from other neighborhoods?
    \item Sub-RQ 4: What are the characteristics of correctly classified signage within the top 20 highest probabilities per class?
    \item Sub-RQ 5: What are the characteristics of misclassified signage?
    \item Sub-RQ 6: What are the characteristics of signage in the extended dataset that are within the top 20 highest probabilities per class?
\end{enumerate}

This paper continues with a description of related work, followed by the study's methodology, results, discussion, and conclusion.