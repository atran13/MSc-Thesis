\section{Discussion}
\label{sec:discussion}
% Compare your results with the state-of-the-art and reflect upon the results and limitations of the study. You can already hint at future work to which you come back in the conclusion section.
% Write your discussion here. Do not forget to use sub-sections. Normally, the discussion starts with comparing your results to other studies as precisely as possible. The limitations should be reflected upon in terms such as reproducibility,  scalability,  generalizability,  reliability  and  validity. It is also important to mention ethical concerns.

\subsection{Summarizing results}
Answering RQs:
1. CRAFT was able to detect signage with reasonable accuracy and completeness, though noise remains.
2. A fine-tuned ResNet50 is able to learn signage features and classify them with 70\% accuracy.
3. There is a gap between the model's performance on StreetSwipe and on extended data of other gentrified/non-gentrified neighborhoods. This suggests that "statistically" gentrified does not necessarily mean visually gentrified. Points to question about the neighborhood's demographic makeup? Maybe displacement did not happen to old residents, or it did happen but old business owners were able to adapt to new residents.
4. Characteristics of signage as classified by the model are in-line with findings of previous research: gentrified signage looks more homogenized, and are more likely to be in English, while non-gentrified signs look more decorated with serif fonts and more colors, and are largely in Dutch.
5. Misclassified signage follow the styles of the class they were classified to (i.e. non-gentrified signage looks like gentrified signage in terms of font and colors)
6. Signage of the extended dataset follow the style of the class they were classified to (they look like the correctly classified instances from StreetSwipe). Suggests model is generalizable to detect gentrification through signage, based on visual perception.

\subsection{Limitations}
- StreetSwipe limitations: (1) Number of votes in pre-july 2020 version is lacking, leads to lower reliability of results, (2) Class imbalance

- Model's performance on gentrified signage - due to class imbalance probs.

- Qualitative studies are able to go into extensive semantic details and point out all the nuances. E.g. for incorrect classifications: can't tell the full story based on visual alone, signage of non-gentrified facades can still appear gentrified and vice versa. The difference must lie in e.g. type of business, and probably other features of the facade.
