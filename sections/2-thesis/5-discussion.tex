\section{Discussion}
\label{sec:discussion}
% Compare your results with the state-of-the-art and reflect upon the results and limitations of the study. You can already hint at future work to which you come back in the conclusion section.
% Write your discussion here. Do not forget to use sub-sections. Normally, the discussion starts with comparing your results to other studies as precisely as possible. The limitations should be reflected upon in terms such as reproducibility,  scalability,  generalizability,  reliability  and  validity. It is also important to mention ethical concerns.

\textit{\textbf{(Still to be written in full)}}

\subsection{Summarizing results}
Answering RQs:

1. CRAFT was able to detect signage with reasonable accuracy and completeness, though noise remains.

2. A fine-tuned ResNet50 is able to learn signage features and classify them with 70\% accuracy.

3. There is a gap between the model's performance on StreetSwipe and on extended data of other gentrified/non-gentrified neighborhoods. This suggests that "statistically" gentrified does not necessarily mean visually gentrified. This points to question about the neighborhood's demographic makeup. It could be the case that displacement did not happen to old residents, or it did happen but old business owners were able to cater to new residents and remained in the gentrified neighborhoods.

4. Characteristics of signage as classified by the model are in-line with findings of previous research: gentrified signage looks more homogenized, and are more likely to be in English, while non-gentrified signs look more decorated with serif fonts and more colors, and are largely in Dutch, with more appearances of other languages.

5. Misclassified signage follow the styles of the class they were classified to (i.e. non-gentrified signage could have minimalist use of fonts and colors, with made them appear gentrified). 

6. Signage of the extended dataset follow the style of the class they were classified to (they look like the correctly classified instances from StreetSwipe). Suggests model is generalizable to detect gentrification through signage, based on visual perception.

\subsection{Limitations}
\subsubsection{Data-related limitations}

- StreetSwipe limitations: (1) Number of votes in pre-july 2020 version is lacking, could lead to lower reliability of results, (2) Class imbalance, (3) Varying dates of the data (dates back to 2009 approx) - while we technically could still learn perception as the images were human-annotated, the results shouldn't be interpreted as fully representing the most up-to-date state of gentrification in Amsterdam.

- Extended data: Text instances generally had much lower resolutions compared to StreetSwipe (bc images were taken from a further distance), therefore could have contributed to model's worsened performance.

\subsubsection{Methodology-related limitations}

- Class imbalance led to model's lower performance on gentrified signage compared to non-gentrified.

- Cropping out text instances loses contextual info: where the text are placed (windows or above the stores' entrances, or standees, posters etc.), text density on signage, numbers of font types and colors used, whether a combination of signage types (above entrance, on window, standee) was used, what the rest of the buildings look like, what else appear in the store facade other than signage (e.g. window displays)

- Traditional qualitative studies are able to go into extensive semantic details and point out all the nuances. E.g. given incorrect classifications in our case: can't tell the full story based on visual alone, signage of non-gentrified facades can still appear gentrified and vice versa, with consistent visual patterns found. The difference must lie in e.g. types of business, locations, and other features of the facade. Such subjective reasonings could be included via using StreetSwipe's mismatch responses, but there was not enough of these to be incorporated into a deep learning model (only 900 responses), plus many were unreliable responses (e.g. respondents putting "X" instead of an actual answer, presumably to be able to skip forward).
