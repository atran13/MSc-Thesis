%%%%%%%%%%%%%%%%%%%%%%%%%%%%%%%%%%%%%%%%%%%%%%%%%%%%%%%%%%%%%%%%%%%%%%%%%%%%%%%%
% setup.tex
% Main tex file for setup and content collocation
% For the use of University of Amsterdam
% Information Systems and Data Science students
% Adapted by Riccardo Fiorista (riccardo.fiorista@proton.me)
%%%%%%%%%%%%%%%%%%%%%%%%%%%%%%%%%%%%%%%%%%%%%%%%%%%%%%%%%%%%%%%%%%%%%%%%%%%%%%%%

% Options:
% Choose one of:
%% `is` - Information Systems
%% `ds` - Data Science 
% Add (separated by `,`):
%% `nolinenumbering` - If you want to remove line numbering on submission
%% `draftmargins` - If you would like to give your reviewer more space for comments
%% `nofrontpicture` - If you do not wish to have a graphic on your front-page
%% `nofirstcompanypicture` - If you do not wish to have a graphic on your front-page
%% `nosecondcompanypicture` - If you do not wish to have a graphic on your front-page
\documentclass[ds, nofrontpicture, nofirstcompanypicture, nosecondcompanypicture]{mscthesis}

%%%%%%%%%%%%%%%%%%%%%%%%%%%%%%%%%%%%%%%%%%%%%%%%%%%%%%%%%%%%%%%%%%%%%%%%%%%%%%%%
% DOCUMENT METADATA
%%%%%%%%%%%%%%%%%%%%%%%%%%%%%%%%%%%%%%%%%%%%%%%%%%%%%%%%%%%%%%%%%%%%%%%%%%%%%%%%

% Thesis related entries
\title{Signs of Gentrification}
\subtitle{An Analysis into Amsterdam's Storefront Signage with Machine Learning and Street View Imagery}

% Date on which your thesis is submitted
\date{DD.MM.YYYY}

% 4-5 keywords should do the trick. They should ideally be phrases of 2-4 words or single words.
\keywords{gentrification, street view imagery, storefront signage, scene-text detection, classification}

% Author data
\authorname{Anh Tran}
\authorid{12770698}
\authoremail{anh.tran1@student.uva.nl}

% Supervisors
\uvasupervisorname{Tim Alpherts}
\uvasupervisoraffiliation{University of Amsterdam}
\uvasupervisoremail{t.o.l.alpherts@uva.nl}

% Comment if you do not have an external supervisor
% \externalsupervisorname{External Supervisor}
% \externalsupervisoraffiliation{External Supervisor}
% \externalsupervisoremail{supervisor@company.nl}

% % Uncomment and fill paths if you want to add custom images
% %% Figure size suggestions (in general it's best to render them from SVGs):
% %% 3000x3000 @ 240dpi for all three
% \titlepicturepath{}
% \firstcompanypicturepath{}
% \secondcompanypicture{path}

%%%%%%%%%%%%%%%%%%%%%%%%%%%%%%%%%%%%%%%%%%%%%%%%%%%%%%%%%%%%%%%%%%%%%%%%%%%%%%%%
% CONTENT
%%%%%%%%%%%%%%%%%%%%%%%%%%%%%%%%%%%%%%%%%%%%%%%%%%%%%%%%%%%%%%%%%%%%%%%%%%%%%%%%

\begin{document}

\pagestyle{plain}
\setcounter{page}{1}

\maketitlepage
\fixemptypage

\title{Signs of Gentrification}
\subtitle{An Analysis into Amsterdam's Storefront Signage with Machine Learning and Street View Imagery}

\begin{abstract}
% % A summary of results should be included. Avoid citations. Maximum length is 200 words.
Gentrification refers to the process of a neighborhood changing as a result of wealthier residents moving in, bringing improvements but displacing existing residents due to rising prices and changing cultures. Studies have pointed out how storefront signage mirrors gentrification: as a rich communication medium with intentional designs, there are clear differences in signage on gentrified and non-gentrified facades. However, these qualitative studies are labor-intensive and are limited in generalizability, as they were conducted on selected neighborhoods, not city-wide. The current research explored the use of computer vision to overcome these limitations in learning signage aesthetics, and simultaneously detect gentrification in signage from unseen parts of the city. Trained on a set of large-scale street view imagery of Amsterdam, the model was able to distinguish gentrified from non-gentrified signage with F1-score of 0.69. The city-wide pattern of gentrification learned by the model was similar to what has been described by past qualitative research. Moreover, the model's output identified cases where signs did not follow the typical patterns - a nuance previous studies did not conclude on. Lastly, the model has the ability to detect the same aesthetics in unseen images of signage from different areas in the city.
\end{abstract}

\begin{teaserfigure}
    \includegraphics[width=0.95\textwidth]{media/titlepage/output_vis-frontman.jpg}
\end{teaserfigure}

\maketitle

\section*{Github Repository}
\url{https://github.com/atran13/MSc-Thesis-Gentrification-and-storefront-signage}

% Sections; Try to stick to this setup but you can comment each section

\section{Introduction}
\label{sec:introduction}
% Mention scientific context/field, problem statement, research gap and candidate (sub) research question(s). 

Within urban studies, gentrification is a phenomenon widely discussed. First coined by British sociologist Ruth Glass in 1964 in her work about the inner city of London \cite{Glass1964}, the term refers to a neighborhood changing as a result of wealthier residents moving in, gradually displacing existing residents as local housing and service prices increase, and cultures homogenized or replaced. Gentrification thus involves an economic and demographic shift, as well as changes in the aesthetics of the built environment. For its negative effects on marginalized communities, it is worthwhile to understand and detect gentrification. This study takes a focus on the visual indicators of gentrification, as visual elements are arguably the most telling factor of a neighborhood’s cultural identity, demographic and economic characteristics.

Gentrification is a multi-dimensional and multi-step process. Döring and Ulbricht \cite{döring_ulbricht_2018} define gentrification as having 4 aspects: functional (establishment of businesses and cultural institutions), architectural (upgrade to the built environment), social (marginalization, displacement and replacement of existing residents), and symbolic (communication of a new image of the neighborhood to the wider public). As has been noted by Feiereisen and Sassin \cite{feiereisen_sounding_2021}, the functional, architectural, and symbolic aspects can be seen as constituting the visual indicators of gentrification. While these three aspects take shape in multiple characteristics of the built environment, on-premise signage, or storefront signage (hereafter: signage) is a rich communication medium that embodies all three aspects. It is through signage that businesses directly establish their presence, communicate their commercial purposes and values, and distinguish themselves via a curated aesthetics \cite{rahman_signage_2020}. Furthermore, businesses understand the socio-cultural values and identity of the neighborhood, and thus design their appearance to best attract and serve this audience:

\begin{displayquote}
    "Shop signs are public texts that communicate what stores sell, who is perceived to be on the street and what their commercial desires are thought to be. [...] Similar to spoken utterances and all written texts, signs are designed for particular audiences [...]. Well-crafted stories are place-making tools inasmuch as they maintain and reproduce prevailing cultural standards and values." \cite{trinch_signsays_2017}
\end{displayquote}

It can therefore be expected that once there is a change in the neighborhood - demographically and economically (i.e. gentrification) - signage would act as a mirror for this change. Analyzing signage can help understand gentrification, and this is where the interest of the current study lies.

Existing research into the visual indicators of gentrification often analyzes elements such as improvements in the neighborhood's physical appearance and changes in architecture style \cite{huang_detecting_2022, ravuri_gsv_2022, naik_computer_2017, ilic_deepmap_2019}. Comparisons have been drawn between gentrified and non-gentrified facades in terms of old versus new features, openness of the properties (e.g. boarded up windows, fences), greenery, colors,...; but not as much attention has been paid to signage as a standalone feature. Most research that has been done in this regards are in the context of the US, in which clear distinctions were made between gentrified and non-gentrified storefront signage \cite{trinch_signsays_2017, snajdr_oldschool_2018, snajdr_preserve_2022, rahman_signage_2020}. The vast majority of these studies on signage employs qualitative methods, whereby the researchers conduct observational data collection (i.e. manually photographing facades) and summarize what is present in their samples. While their findings provide invaluable insights, their methodologies are undoubtedly labor intensive. Furthermore, as has been pointed out by Reades et al. \cite{reades_understanding_2019} and Barton \cite{barton_exploration_2016}, such selection of neighborhoods and facades per neighborhood often comes with limitations in terms of generalizability on a city-wide scale.

Concurrently, the increasing availability of street view data and machine learning techniques has led to developments in Urban Visual Intelligence \cite{zhanga_urban_2023}, whereby cities' built environments are understood in conjunction with socio-economic circumstances and residents' activities on a large scale. Besides predictive modelling for gentrification based on historic socio-economic transitions \cite{thackway_build_2021, reades_understanding_2019}, work has been done to visually measure gentrification via documenting changes \cite{ravuri_gsv_2022}, detect \cite{huang_detecting_2022} and deep-map gentrification to reveal areas unknowingly becoming gentrified \cite{ilic_deepmap_2019}. A small number of machine learning research took a focus on signage, but in terms of linguistic landscape \cite{hong_linguistic_2020, thung_detecting_2022}, typeface (font type) \cite{ma_typef_2019}, or to classify points of interest \cite{noorian_detect_2020, bakaev_stsem_2019}, instead of to understand the overall aesthetic of signage.

Systematic literature reviews on urban analytics \cite{biljecki_street_2021, zhanga_urban_2023} have noted the applicability and increasing importance of the approach in generating insights and decision-making, while pointing future research to further analyzing written languages in images, as well as between-place inference (applying a machine learning model trained with one area to another area). The current study positions itself in this research gap, where it aims to expand the knowledge of signage aesthetics as a mirror of gentrification in Amsterdam, while utilizing computer vision and street view imagery to overcome limitations of existing urban ethnographic and socio-economic studies.

The dataset at the center of this research is from the StreetSwipe project. Using crowdsourcing, StreetSwipe \cite{streetswipe} lets people decide whether each facade in Amsterdam appears gentrified. By using this data, this study drew conclusions based on the subjective and common perception of a diverse group of people - arguably a necessity when it comes to understanding a nuanced and multi-faceted phenomenon such as gentrification. Moreover, with the images being sampled from locations all over the city, the results were not constrained per neighborhoods. Understanding what people see as gentrified not only means understanding the current state of gentrification, but also would help detect potential areas undergoing gentrification. As a pilot case, the main goal of the study was to see to which extent a computer vision model can learn the visual perception and, correspondingly, classify signage as perceptually gentrified or non-gentrified. Subsequently, with added data from more areas of the city that was not covered in StreetSwipe, the model was tested to quantify the extent to which this perception hold against the actual state of the neighborhoods (i.e. given all signage from a gentrified neighborhood (as per census data), how many of the signage would be visually perceived as gentrified?). Lastly, via inspecting the model's predictions, insights are provided into the characteristics of gentrified and non-gentrified signage in Amsterdam, as well as cases where the distinction was less clear.

The research question of this study was stated as follow: 

\noindent\textit{To which extent can we learn the visual differences between perceptually gentrified and non-gentrified storefront signage using a computer vision model? How does this visual perception generalize given data from other neighborhoods of the city? Lastly, what are the defining characteristics that distinguish gentrified from non-gentrified signage, and which characteristics make the distinction less clear?}

To answer this research question, this study clarified the following sub-questions: 

\begin{enumerate}
    \item Sub-RQ 1: To which extent can the scene-text detection model CRAFT identify storefront signage from street view images?
    \item Sub-RQ 2: To which extent can fine-tuned ResNets correctly identify gentrified and non-gentrified signage?
    \item Sub-RQ 3: How does the model trained on visual perception perform on a test set of data from other neighborhoods, labeled per census data?
    \item Sub-RQ 4: What are the characteristics of correctly classified StreetSwipe signage per class with classification probability of 80\% and above?
    \item Sub-RQ 5: What are the characteristics of mis-classified StreetSwipe signage with high ($ \geq 80\% $) and low (50-70\%) classification probability?
    \item Sub-RQ 6: What are the characteristics of signage per class in the extended dataset with classification probability of 80\% and above (irregardless of the ground truth)?
\end{enumerate}

This paper continues with a description of related works, followed by the study's methodology, results, discussion, and conclusion.
\section{Related Work}
\label{sec:related_work}

This section first outlines the findings of current ethnographic and urban studies on signage and gentrification, followed by a brief exploration into scene-text attribute learning. Next, the state-of-the-art of scene-text detection are given. Finally, image classification is discussed and the computer vision model to be used in the study is presented.

\subsection{Signage and gentrification}

As aforementioned, the current body of research about storefront signage aesthetics in relation to gentrification exists largely in the context of the US. While some studies took a holistic approach and considered multiple elements, namely font type (or typeface), colors, text density, language, and meaning, others analyzed elements individually.

Findings from Brooklyn, New York\cite{trinch_signsays_2017, snajdr_oldschool_2018, snajdr_preserve_2022} show that non-gentrified signage - or as the authors call them: \textit{old-school} signage - typically is text-dense and has larger typeface; names that refer to the location, the owner's name, the type of business, products or services; languages other than English; complementary symbols or images; reference to religion, ethnicity, country of origin, and race. On the contrary, gentrified signage - or \textit{distinction-making} signage - has shorter texts, written in smaller font sizes, lower case letters; more cryptic or ambiguous names, sometimes polysemic and with word-plays; languages other than English that shows sophistication and worldliness. In parallel, a study from Cincinnati, Ohio \cite{rahman_signage_2020} found that signage forms and characters became more homogeneous as neighborhoods became gentrified.

Next to this, other studies have also analyzed storefront signage but with their focus on individual attributes. A popular topic within gentrification in this regard is the linguistic landscape of a city or neighborhood. Examples can be found in Seoul, South Korea \cite{hong_linguistic_2020}: in the neighborhood where the majority of the Chinese population in Seoul reside, Korean signage has been gradually replaced by Chinese signage, mirroring both the demographic composition and the social and economic standing of these groups. Similarly, more English signage has appeared in a commercial neighborhood in Phnom Penh, Cambodia, seemingly displacing French as a second language - a trend observed alongside globalization, gentrification, a generational change in attitudes, and education policies \cite{kasanga_map_2012}. Besides languages in signage, typeface has also been found to be highly correlated with average household income of the corresponding neighborhoods in London \cite{ma_typef_2019}.

All in all, findings show clear differences between the characteristics of gentrified and non-gentrified signage. It was therefore hypothesized that the same pattern can be found in Amsterdam's signage, and thus a computer vision model can be used to distinguish and detect gentrification based on signage.

\subsubsection{Scene-text attribute learning}
Taking inspiration from the insights outlined above, where signage was analyzed in terms of typeface, color, and textual meaning (among other elements), an exploration in this direction was done for the current study. The initial goal was to learn these specific attributes and quantify, on a large scale, what characteristics differentiate gentrified and non-gentrified signage in Amsterdam. The pipeline would involve detecting the text instances from the original images (text detection), classifying the font type, inferring the dominant colors, transcribing the text (text recognition), and learn semantics via word embeddings. However, given the state of the dataset at hand - namely, the lack of annotations other than the gentrification label - as well as resources available, this research direction was ultimately deemed unsuitable. The state-of-the-art approaches are described below.

\begin{enumerate}
    \item End-to-end: These are models that can take an image and output multiple elements of interest. Examples are NeurTEx \cite{aggarwal_neurtex_2022} (outputs text bounding box, transcript, and typeface), Fontnet \cite{s_fontnet_2021} (outputs color and typeface), and TaCo \cite{nie_taco_2022} (outputs color and typeface). However, these models are designed to be applied in the context of graphic design and on documents. They are not reliable when it comes to scene-text because of the added noise and distortions of natural images.
    \item Step-by-step: This approach uses individual models to learn each element:
    \begin{itemize}
        \item Font: State-of-the-art models include DeepFont \cite{wang_deepfont_2015}, HENet \cite{chen_henet_2021}, and benchmark datasets include AdobeVFR \cite{wang_deepfont_2015}, and VFR-2420 \cite{chen_large-scale_2014}. These data are largely synthetic, and accordingly the models perform well on synthetic data, but are not as robust when it comes to natural images. Furthermore, with neither pre-trained models nor ground truth available on StreetSwipe images, training and evaluation proves to be challenging and unreliable.
        \item Text transcripts and semantics: State-of-the-art scene-text recognition models include MORAN \cite{luo_multi-object_2019}, CRNN \cite{shi_end2end_2015}, and PARSeq \cite{bautista_scene_2022}. For learning semantics, word embedding models such as FastText \cite{bojanowski_enriching_2017} or Word2Vec \cite{mikolov_efficient_2013} could be applied on the transcribed text. However, as was found from text detection (discussed in a later section), the text instances from the data were either not meaningful words (a nature of signage text), or fragmented upon detection (words are broken up). Learning semantics from this data would not give meaningful and consistent results.
        \item Color: Text colors as a standalone feature could be learnt via creating a color histogram \cite{srivastava_review_2015} per image. But instead, by using a convolution neural network (discussed in a later section), colors can be learned as well along with many other features.
    \end{itemize}
\end{enumerate}

To account for the lack of ground truth labels, it was also considered that a synthetic scene-text dataset is created to be used as training and testing data, before inference is done on StreetSwipe. Gupta et al. \cite{gupta_synt_2016} offers a method for this that takes into account image segmentation and depth in order to generate text, with annotations on bounding box, typeface, color, and text transcript as needed. An example usage in a closely related study was done by Ma et al. \cite{ma_typef_2019}, in which the authors generated training data to recognize typeface on signage. Considering time constraint, however, this approach was non-viable. It required either mining street view images that do not have text to be synthesized on, or using pre-generated images from Gupta et al., which largely included many types of background other than street view of facades. Either method proved impractical - the former is time-consuming, even when discounting for model training and tuning; and the latter is a domain mismatch compared to StreetSwipe.

All in all, a machine learning approach for replicating existing studies was not found. A more appropriate research direction was devised that involved learning a more general visual representation of the signage with a convolutional neural network. Signs would firstly be extracted using a pre-trained text-detection model, and a classification model would be trained and tested on these signage.

\subsection{Scene-text detection}
Scene-text detection involves identifying and localizing text in natural images - a task fundamental to the current study. The challenge in this task stems from extracting text from complex images, with the text surrounded by other objects, varying in sizes, perspectives, orientations, sometimes curved, obstructed, or blurry.

Benchmark datasets for this task include ICDAR 2013 \cite{icdar13} and 2015 \cite{icdar15}, and TotalText \cite{totaltext}. The most widely implemented models are EAST \cite{zhou_east_2017} and CRAFT \cite{baek_character_2019}. CRAFT is the best performing model on ICDAR 2013, and surpasses EAST on ICDAR 2015 and TotalText. CRAFT is more accurate on curved, long, and non-horizontal text. By calculating character region scores (localizing characters) and affinity scores between characters (grouping characters into sequences), the model returns word-level bounding boxes. The model is also multilingual - a necessary feature considering signage in the dataset at hand are (at least) in Dutch, English, Chinese and Korean. CRAFT is thus the model used in this study for extracting signage.

The pre-trained CRAFT model was implemented via the EasyOCR Python package \cite{noauthor_jaided_nodate}, which offered an ease of implementation and also allowed for processing multiple languages simultaneously.

\subsection{Image classification}
The Residual Network (ResNet) \cite{resnet}, with the use of skip connections in its architecture, has enabled deeper networks to learn more efficiently, without the problem of vanishing or exploding gradients. For the task of learning and classifying signage, fine-tuned ResNet18 and ResNet50 were chosen as candidate models, initialized with pre-trained weights from ImageNet. After achieving satisfactory performance, the best model's predictions were further analyzed. 

Firstly, the correctly classified signage within the top 20 highest probabilities from each class were inspected - this showed the most typical and distinguishing characteristics of gentrified and non-gentrified signage. In comparison with existing studies (as outlined in section 2.1), conclusions were drawn on the extent to which patterns found elsewhere were similar to our case of Amsterdam.

Secondly, the incorrectly classified signage were inspected. These can be considered the more fuzzy cases - something that previous studies did not report on.

Finally, the model was tested on an external dataset of other neighborhoods in the city. Given literature on gentrification in Amsterdam, gentrified and non-gentrified neighborhoods were selected, and street view images from these areas were retrieved from a dataset made available by the Civic AI Lab. Gentrified areas include Jordaan \cite{verlaan_hippies_2022}, Oud-West, De Baarjes \cite{rettberg_when_2019}; and non-gentrified areas include Zuidoost and Nieuw-West \cite{pinkster_stickiness_2020}. Images taken from these neighborhoods are labeled accordingly. Thus, when testing the model (trained on visual perception) on this data (labeled per census data), we could tell to which extent the perception is applicable on a neighborhood level. In other words, given a gentrified neighborhood, to which extent are signage in that neighborhood seen as gentrified? Further, the model's predictions on this data were also visualized as per the highest 20 probabilities per class, irrespective of ground truth labels. This showed what the model considered as gentrified and non-gentrified, and were this to be in-line with characteristics found in StreetSwipe, this would support the model's generalizability in detecting gentrification.


\section{Methodology}
\label{sec:methodology}
% Focus on what you add to the existing method. Explain what you will do and why (and how). Do not forget to characterize your research design. There should be a sub-section on the evaluation. 
%For DS students, this normally means using manually labelled or ground truth data. 
% Write about your methodology here. Focus on your own contribution. Indicate exactly how you will assess your work in terms of evaluation.
% It is possible to use a separate section for the Experimental Setup, which then focuses on all settings used in your experiments. It also possible to address the settings in a sub-section under Methodology. 

The goal of the study is to understand the characteristics of storefront signage in relation to gentrification, via learning the font types, colors, and semantics present. The approach that this study takes comprises of multiple models: scene-text detection, scene-text recognition, color histogram, font recognition, word embedding, and gradient boosting. This is because the state-of-the-art is that there has not been a model that takes an interest in analyzing scene-text in a comprehensive manner - much of the focus is on detecting [CITE] and recognizing [CITE] text in the wild with better accuracy. Elements such as fonts and colors are only of interest in the domain of graphic design [CITE]. Therefore, this research serves as a pilot case in utilizing these individual models for feature extraction via transfer learning, and subsequently test these features by using them to classify gentrified/non-gentrified storefronts.

The data pipeline is visualized in Figure \ref{fig:pipeline}.

\begin{figure*}[]
    \centering
    \includegraphics[width=\textwidth]{media/methodology/Pipeline.jpg}
    \caption{Pipeline: Images of facades labelled gentrified or non-gentrified are first fed into the text detection model to extract text instances. Subsequently, the text fonts are extracted with the font classifier, and color histograms are made to record the colors present in each text instances. Text strings are extracted with the text recognition model, before being converted into a vector embedding. \textbf{\textit{[A binary classifier]}} is trained and tested on these 3 features, to classify gentrified facades.}
    \label{fig:pipeline}
\end{figure*}

\subsection{Data}
\subsubsection{StreetSwipe}
The dataset with gentrified and non-gentrified labels is retrieved from the StreetSwipe project \cite{streetswipe}. Using crowd-sourcing, the project let people decide whether each Amsterdam facade is gentrified, by voting "Yes" or "No" on the street view images of these facades. The official \textit{Gentrified} and \textit{Non-gentrified} labels are based on what the majority of people voted for, for each facade. Additionally, if subsequent voters decide against the majority (e.g. voting \textit{Gentrified} for a non-gentrified-labelled facade), they are also prompted to provide a textual reasoning for their decision. These mismatch responses are also available, however is out of scope of the study. 

Since there are two versions of StreetSwipe, the data retrieved exists in two sets, consisting of 1912 higher resolution images from the older version and 529 lower resolution images from the new one. The StreetSwipe dataset thus have 2,441 images in total, each with its numbers of "Yes" and "No" votes, and metadata on the facade's location (latitude and longitude) and street name. The new version's images also have more detailed address, name and type of business/services. There are also more votes in the new version than in the older version. The images from the old version are available directly, while the new ones were provided via URLs to a Google APIs bucket, and thus were scraped.

Feature engineering was done to create the gentrified/non-gentrified label per image, by taking the vote with higher number ("Yes" for gentrified, "No" for non-gentrified). The images were then re-grouped per their corresponding label. Figure \ref{fig:class_size_SS} shows the sample size per class. There is class imbalance in the data, with more than 70\% of the images labelled non-gentrified. This is accounted for in evaluation by using appropriate metrics for classification performance.

\begin{figure}[H]
    \centering
    \includegraphics[width=0.4\textwidth]{media/methodology/SS_class_size.png}
    \caption{Sample size per class in StreetSwipe.}
    \label{fig:SS_class_size}
\end{figure}

\begin{figure}[H]
    \centering
    \includegraphics[width=0.35\textwidth]{media/methodology/SS_size_ar.png}
    \caption{Image sizes and aspect ratio distribution in StreetSwipe.}
    \label{fig:SS_size_ar}
\end{figure}

Figure \ref{fig:SS_size_ar} shows that the images have quite consistent aspect ratios of approximately 1:1; however, they vary in size, ranging from around 300x300 to 1700x1700, with one outlier of size 2500x1300 (approximately). No normalization was done at this stage, as the text detection framework does not require a specific image size.

While this dataset contains valuable information - that is crowd-sourced label on gentrification - its size is sub-optimal for training a machine learning model. Therefore, it is used as the test set for the classifier, and another, larger data is used for training.

\subsubsection{Amsterdam street view data}


\subsection{Experimental setup}
\subsubsection{Scene-text detection - CRAFT}
Detecting text in an image, by creating bounding boxes using CRAFT (Character-Region Awareness For Text detection). Compared to another state-of-the-art text detection model - EAST (Efficient and accurate scene text detector) - CRAFT is more accurate and is multi-lingual.

\subsubsection{Scene-text recognition - PARSeq}
Transcribe the image-text into text strings readable by models.

\subsubsection{Color detection}
Create color histograms for the distribution of colors present in the text box.

\subsubsection{Typeface recognition - DeepFont} 

\subsubsection{Semantic analysis - FastText}
Transform the extracted texts into a word vector representation using FastText, to recognize languages and semantics such as cryptic or literal store names, inclusion of ethnicity, religions, etc.

\subsection{Evaluation}
Evaluation is done with the goal to see how accurate the models for extracting fonts, colors, and semantic are. For font and color, this is aimed to be done with the Ma et al.'s synthetic data, with an 80-20 split for the train and test set, respectively. It is appropriate as it has ground truth labels for font and color of text, whereby precision and recall will be calculated. As for the word embedding model, since this is unsupervised, evaluation will be done by examining the word vector's nearest neighbors to see the semantic information the vector captured.
\section{Results}
\label{sec:results}

\subsection{Scene text detection}
\label{subsec:result1}
Some example signage instances per class can be seen in Figure \ref{fig:instance_ex}.

{
\setlength\intextsep{7pt}
\begin{figure}[H]
\centering
\begin{subfigure}[b]{0.2\textwidth}
    \includegraphics[width=\textwidth]{media/results/instances/instance_gen.jpg}
    \caption{Gentrified}
\end{subfigure}
\quad
\begin{subfigure}[b]{0.22\textwidth}
    \includegraphics[width=\textwidth]{media/results/instances/instance_non.jpg}
    \caption{Non-gentrified}
\end{subfigure}
\caption{Signage instances examples.}
\label{fig:instance_ex}
\end{figure}
}

It was found that the text detection model returned almost all signage instances present in the original street view images, including signage with non-horizontal and curved text. Cases where the model failed include very small, and therefore illegible texts, especially in lower resolution images. Additionally, the model also returned some noise, namely texts on street signs (e.g. traffic signs, street names), and watermarks ("©Google", as the images in Street-Swipe were originally from Google Street View) - these instances were manually removed.


\subsection{Classification}
\subsubsection{Baseline}
Test set results of the baseline model are shown in Table \ref{fig:baseline_metrics}. 

% {
% \setlength\intextsep{7pt}
\begin{table}[h]
\begin{tabular}{llcc}
\toprule
\multirow{2}{*}{Metrics}   & \multirow{2}{*}{Class} & \multicolumn{2}{c}{Baseline model}        \\ \cline{3-4} 
                           &                        & Classwise & Average                 \\ \hline
Accuracy                   & Gentrified             & 0.2143    & \multirow{2}{*}{0.5810} \\
                           & Non-gentrified         & 0.9478    &                         \\
Precision                  & Gentrified             & 0.6122    & \multirow{2}{*}{0.6852} \\
                           & Non-gentrified         & 0.7582    &                         \\
Recall                     & Gentrified             & 0.2143    & \multirow{2}{*}{0.5810} \\
                           & Non-gentrified         & 0.9478    &                         \\
F1-score                   & Gentrified             & 0.3175    & \multirow{2}{*}{0.5800} \\
                           & Non-gentrified         & 0.8425    &                         \\
\bottomrule
\end{tabular}
\vspace{\baselineskip}
\caption{Classwise and macro-averaged test set metrics of baseline model (ResNet50, no weighted sampling, no fine-tuning). This model achieved very high performance for non-gentrified signage, but performed poorly on gentrified signage, due to class imbalance.}
\label{fig:baseline_metrics}
\vspace{-5mm}
\end{table}
% }

\subsubsection{Fine-tuned}
The best performing model with ResNet18 architecture was found with learning rate set to 0.001, batch size 32, and 60 training epochs. The best performing model with ResNet50 architecture was found with learning rate set to 0.01, batch size 64, and 60 training epochs. The macro-averaged metrics for these models on the validation and test sets are shown in Table \ref{fig:resnet_compare}. The fine-tuned ResNet50 had better performance, its classwise metrics are shown in Table \ref{fig:resnet50_cls}.

% {
% \setlength\intextsep{10pt}
\begin{table}[h!]
    \begin{tabular}{lcccc}
    \toprule
\multirow{2}{*}{Metrics} & \multicolumn{2}{c}{ResNet18} & \multicolumn{2}{c}{ResNet50} \\ \cmidrule(lr){2-3} \cmidrule(lr){4-5}
                         & Val           & Test          & Val           & Test         \\ \hline
Accuracy                 & 0.6497        & 0.6960        & 0.6506        & \textbf{0.7033}       \\
Precision                & 0.6185        & 0.6715        & 0.6209        & \textbf{0.6795}       \\
Recall                   & 0.6497        & 0.6960        & 0.6506        & \textbf{0.7033}       \\
F1-score                 & 0.6222        & 0.6781        & 0.6256        & \textbf{0.6865}       \\ \bottomrule
    \end{tabular}
    \vspace{\baselineskip}
    \caption{Macro-averaged metrics of fine-tuned ResNet18 and ResNet50 on the validation and test set. Between these two models, the fine-tuned ResNet50 performed better.}
    \label{fig:resnet_compare}
    \vspace{-5mm}
\end{table}
% }

% {
% \setlength\intextsep{0pt}
\begin{table}[h!]
\begin{tabular}{llcc}
\toprule
\multirow{2}{*}{Metrics}   & \multirow{2}{*}{Class} & \multicolumn{2}{c}{ResNet50} \\ \cline{3-4} 
                           &                        & Val           & Test         \\ \hline
Accuracy                   & Gentrified             & 0.5714        & 0.6429       \\
                           & Non-gentrified         & 0.7299        & 0.7637       \\
Precision                  & Gentrified             & 0.3953        & 0.5114       \\
                           & Non-gentrified         & 0.8464        & 0.8476       \\
Recall                     & Gentrified             & 0.5714        & 0.6429       \\
                           & Non-gentrified         & 0.7299        & 0.7637       \\
F1-score                   & Gentrified             & 0.4674        & \textbf{0.5696}       \\
                           & Non-gentrified         & 0.7838        & \textbf{0.8035}       \\
\bottomrule
\end{tabular}
\vspace{\baselineskip}
\caption{Classwise metrics of the best model. Even though there was an improvement in classifying gentrified signage compared to the baseline model, this model still performed better for non-gentrified signage, as shown in the F1-scores of each class.}
\label{fig:resnet50_cls}
\vspace{-7mm}
\end{table}
% }

\subsection{Testing on the extended data}
The average and classwise metrics of the best model in classifying the extended data are presented in Table \ref{fig:resnet50_pano}.

% {
% \setlength\intextsep{2.65pt}
\begin{table}[h!]
\begin{tabular}{llcc}
\toprule
\multirow{2}{*}{Metrics}   & \multirow{2}{*}{Class} & \multicolumn{2}{c}{ResNet50}   \\ \cline{3-4} 
                           &                        & Classwise & Average                 \\ \hline
Accuracy                   & Gentrified             & 0.5340    & \multirow{2}{*}{0.5807} \\
                           & Non-gentrified         & 0.6274    &                         \\
Precision                  & Gentrified             & 0.7359    & \multirow{2}{*}{0.5725} \\
                           & Non-gentrified         & 0.4092    &                         \\
Recall                     & Gentrified             & 0.5340    & \multirow{2}{*}{0.5807} \\
                           & Non-gentrified         & 0.6274    &                         \\
F1-score                   & Gentrified             & 0.6189    & \multirow{2}{*}{0.5571} \\
                           & Non-gentrified         & 0.4953    &                         \\
\bottomrule
\end{tabular}
\vspace{\baselineskip}
\caption{Macro-averaged and classwise metrics of the best model on the extended data. Compared to the metrics on StreetSwipe's test set (Table \ref{fig:resnet_compare}), there was a decrease of approximately 10\% in all average metrics.}
\label{fig:resnet50_pano}
\vspace{-7mm}
\end{table}
% }

\subsection{Inspecting model's output}
\label{subsec:result4}

\subsubsection{Correct classifications}

StreetSwipe's correctly classified signage showed the most typical and distinguishing characteristics of signage per class, which can be seen in Figure \ref{fig:output_vis}.

Gentrified signage were more similar in font types (more modern and minimal fonts) and often did not vary in text sizes, while non-gentrified signage used more types of fonts (more classic and decorated fonts), sometimes more than one font on a single sign, and sometimes with varying text sizes and orientations. In addition, gentrified signs mostly had white texts, with minimal variation in background colors. Non-gentrified signs were the opposite: text colors and background colors varied more; plus a notable usage of neon signage. In terms of languages, besides Dutch, gentrified signage had more English text, with very rare appearances of non-Latin languages (e.g. Korean); while non-gentrified ones were largely in Dutch, with appearances of English, Chinese and Arabic. And finally, although out of scope of the study, the model also picked up graffiti as text instances belonging to non-gentrified facades.


\subsubsection{Incorrect classifications} 

StreetSwipe's misclassified signage showed characteristics of cases the model failed to distinguish. The misclassified instances with high certainty showed similar characteristics to correctly classified instances (gentrified: minimal text fonts and colors, less variation in font styles and background colors; non-gentrified: more variation in fonts, text colors and background colors). On the other hand, mis-classified instances with lower certainty showed a more nuanced picture. Signal strength diminished and the model's outputs of the two classes were more or less indistinguishable. There were similar variations in font styles, text colors and background colors, without any characteristic that stood out. Results can be seen in Figure \ref{fig:output_vis_SS_incorrect}.


\subsubsection{Classifications on extended data}

On the extended data, signage classified by the model followed the same patterns per class as learned from StreetSwipe; however, visual inspection of the corresponding facades showed not all classifications were true to human perception. Results can be seen in Figure \ref{fig:output_vis_pano}.

% {
% \setlength{\floatsep}{0pt}
\begin{figure*}[hbtp]
    \centering
    \includegraphics[width=\textwidth]{media/results/output_vis-SS_correct1.jpg}
        \caption{StreetSwipe's correctly classified signage per class with probability of 80\% and above. Note how non-gentrified signage varied more in their characteristics (more font types, colors, and languages) while gentrified signage appeared more homogenized.}
        \label{fig:output_vis}
    \vspace{0.5cm}
    \includegraphics[width=0.8\textwidth]{media/results/output_vis-SS_incorrect.jpg}
        \caption{StreetSwipe's misclassified signage per class, grouped by high and low classification certainty. Incorrect classifications with high certainty from both classes generally had the same characteristics as correctly classified instances. As classification certainty decreased, variations in fonts and colors were no longer distinctive across the two classes.}
        \label{fig:output_vis_SS_incorrect}
    \vspace{0.6cm}
    \includegraphics[width=\textwidth]{media/results/output_vis_pano.jpg}
        \caption{Model's classifications on the extended data's signage with classification probability of 80\% and above (disregarding ground truth label). While the model classified signage based on the same typical class characteristics as in StreetSwipe, classifications were not always plausible: Bánh Mì Deli (left) and Personality (right) were gentrified-looking facades, but Personality was misclassified as non-gentrified via its signage.}
        \label{fig:output_vis_pano}
\end{figure*}
% }

\section{Discussion}
\label{sec:discussion}
% Compare your results with the state-of-the-art and reflect upon the results and limitations of the study. You can already hint at future work to which you come back in the conclusion section.
% Write your discussion here. Do not forget to use sub-sections. Normally, the discussion starts with comparing your results to other studies as precisely as possible. The limitations should be reflected upon in terms such as reproducibility,  scalability,  generalizability,  reliability  and  validity. It is also important to mention ethical concerns.

\textit{\textbf{(Still to be written in full)}}

\subsection{Summarizing results}
Answering RQs:

1. CRAFT was able to detect signage with reasonable accuracy and completeness, though noise remains.

2. A fine-tuned ResNet50 is able to learn signage features and classify them with 70\% accuracy.

3. There is a gap between the model's performance on StreetSwipe and on extended data of other gentrified/non-gentrified neighborhoods. This suggests that "statistically" gentrified does not necessarily mean visually gentrified. This points to question about the neighborhood's demographic makeup. It could be the case that displacement did not happen to old residents, or it did happen but old business owners were able to cater to new residents and remained in the gentrified neighborhoods.

4. Characteristics of signage as classified by the model are in-line with findings of previous research: gentrified signage looks more homogenized, and are more likely to be in English, while non-gentrified signs look more decorated with serif fonts and more colors, and are largely in Dutch, with more appearances of other languages.

5. Misclassified signage follow the styles of the class they were classified to (i.e. non-gentrified signage could have minimalist use of fonts and colors, with made them appear gentrified). 

6. Signage of the extended dataset follow the style of the class they were classified to (they look like the correctly classified instances from StreetSwipe). Suggests model is generalizable to detect gentrification through signage, based on visual perception.

\subsection{Limitations}
\subsubsection{Data-related limitations}

- StreetSwipe limitations: (1) Number of votes in pre-july 2020 version is lacking, could lead to lower reliability of results, (2) Class imbalance, (3) Varying dates of the data (dates back to 2009 approx) - while we technically could still learn perception as the images were human-annotated, the results shouldn't be interpreted as fully representing the most up-to-date state of gentrification in Amsterdam.

- Extended data: Text instances generally had much lower resolutions compared to StreetSwipe (bc images were taken from a further distance), therefore could have contributed to model's worsened performance.

\subsubsection{Methodology-related limitations}

- Class imbalance led to model's lower performance on gentrified signage compared to non-gentrified.

- Cropping out text instances loses contextual info: where the text are placed (windows or above the stores' entrances, or standees, posters etc.), text density on signage, numbers of font types and colors used, whether a combination of signage types (above entrance, on window, standee) was used, what the rest of the buildings look like, what else appear in the store facade other than signage (e.g. window displays)

- Traditional qualitative studies are able to go into extensive semantic details and point out all the nuances. E.g. given incorrect classifications in our case: can't tell the full story based on visual alone, signage of non-gentrified facades can still appear gentrified and vice versa, with consistent visual patterns found. The difference must lie in e.g. types of business, locations, and other features of the facade. Such subjective reasonings could be included via using StreetSwipe's mismatch responses, but there was not enough of these to be incorporated into a deep learning model (only 900 responses), plus many were unreliable responses (e.g. respondents putting "X" instead of an actual answer, presumably to be able to skip forward).

\section{Conclusion}
\label{sec:conclusion}

This study set out to study storefront signage aesthetics in relation to gentrification, utilizing street view image data and computer vision for a larger-scale analysis and better generalizability. By fine-tuning ResNets on 
the StreetSwipe dataset and analyzing the model output, conclusions to the research questions are as follow:

\begin{enumerate}
    \item The scene-text detection model CRAFT was able to detect signage with reasonable accuracy and completeness, though noise remains in the form of unwanted text (e.g. traffic signs).
    
    \item A fine-tuned ResNet50 was able to learn signage features and classify them with a macro-average F1 score of 0.69.
    
    \item There was a gap of approximately 10\% between the model's performance on Street-Swipe and on extended data of other gentrified/non-gentrified neighborhoods. This suggested that "statistically" gentrified did not entirely mean visually gentrified. 
    
    \item Characteristics of signage correctly classified by the model were in-line with findings of previous research: gentrified signage looked more homogenized, and were more likely to be in English, while non-gentrified signs varied more in fonts and colors, and were largely in Dutch, with more appearances of other languages.
    
    \item Mis-classified signage with high certainty followed the styles of the class they were assigned to. As classification certainty decreased, the distinction diminished.
    
    \item Signage classifications in the extended dataset follow the same characteristics learned from StreetSwipe. However, inspection of corresponding facades showed that the model was not always giving a plausible detection.
\end{enumerate}

In general, it was found that a computer vision model could do a reasonable job at identifying typical characteristics of gentrified and non-gentrified signage. Further, the model is able to detect the same characteristics on new data, and point out the nuance between the visual and socio-economic states of gentrification. Nonetheless, there were cases where the model failed to detect gentrification based on signage alone. When signage adopted characteristics of the opposite class, perception of gentrification most likely depended on other visual and non-visual features of the facade. While this study was able to show the potential of computer vision in studying signage gentrification, further work is needed in this regard. Expanding the feature space beyond visual signals can be promising in aiding computer vision models to understand the nuances of human perception.

\bibliographystyle{ACM-Reference-Format}
\bibliography{bibliographies/references}

\newpage
% You can choose whether you prefer a single or double column appendix.
% Whatever you choose, you will need to stick to it throughout the appendix.
% For double column style, comment the next line.
\onecolumn

\appendix
\begin{appendices}

\section{Extended data example images.}
\label{sec:apx:appendix1}

\begin{figure}[H]
    \centering
    \includegraphics[width=0.98\textwidth]{media/methodology/data_ex/extended/pano_example.jpg}
    \caption{Example of the extended data.}
    % \label{fig:pano_example}
\end{figure}

\section{Text instance sizes and aspect ratios}
\label{sec:apx:appendix2}

\begin{figure}[H]
    \centering
    \includegraphics[width=\textwidth]{media/methodology/SS_ins_sz.png}
    \caption{StreetSwipe text instance size and aspect ratio}
    \label{fig:SS_ins_sz}
\end{figure}

\begin{figure}[H]
    \centering
    \includegraphics[width=\textwidth]{media/methodology/pano_ins_sz.png}
    \caption{Extended data text instance size and aspect ratio}
    \label{fig:pano_ins_sz}
\end{figure}


\end{appendices}


\end{document}

%%%%%%%%%%%%%%%%%%%%%%%%%%%%%%%%%%%%%%%%%%%%%%%%%%%%%%%%%%%%%%%%%%%%%%%%%%%%%%%%
%%%%%%%%%%%%%%%%%%%%%%%%%%%%%%%%%%%%%%%%%%%%%%%%%%%%%%%%%%%%%%%%%%%%%%%%%%%%%%%%