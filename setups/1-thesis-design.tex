%%%%%%%%%%%%%%%%%%%%%%%%%%%%%%%%%%%%%%%%%%%%%%%%%%%%%%%%%%%%%%%%%%%%%%%%%%%%%%%%
% setup.tex
% Main tex file for setup and content collocation
% For the use of University of Amsterdam
% Information Systems and Data Science students
% Adapted by Riccardo Fiorista (riccardo.fiorista@proton.me)
%%%%%%%%%%%%%%%%%%%%%%%%%%%%%%%%%%%%%%%%%%%%%%%%%%%%%%%%%%%%%%%%%%%%%%%%%%%%%%%%

% Options:
%% `nolinenumbering` - If you want to remove line numbering on submission
%% `draftmargins` - If you would like to give your reviewer more space for comments
\documentclass[nolinenumbering]{thesisdesign}

%%%%%%%%%%%%%%%%%%%%%%%%%%%%%%%%%%%%%%%%%%%%%%%%%%%%%%%%%%%%%%%%%%%%%%%%%%%%%%%%
% DOCUMENT METADATA
%%%%%%%%%%%%%%%%%%%%%%%%%%%%%%%%%%%%%%%%%%%%%%%%%%%%%%%%%%%%%%%%%%%%%%%%%%%%%%%%

% Thesis related entries
\title{Gentrification in Storefront Signage in Amsterdam: A Machine Learning Approach with Streetview Imagery}
\subtitle{Submitted on: \textbf{19-02-2023}}

% Author data
\authorname{Author: Anh Tran}
\authoremail{anh.tran1@student.uva.nl}
% ACM-sepecific entries
\affiliation{
    \institution{\thesisinstitution}
    \city{\thesiscity}
    \country{\thesiscountry}
}

% Supervisor data
\authorname{Supervisor: Tim Alpherts}
\authoremail{t.o.l.alpherts@uva.nl}
% ACM-sepecific entries
\affiliation{
    \institution{\thesisinstitution}
    \city{\thesiscity}
    \country{\thesiscountry}
}

\begin{abstract}
    Gentrification refers to the process of a neighborhood changing as a result of wealthier residents moving in, bringing investments and improvements, but displacing the existing residents due to rising prices and changing cultures. More recent developments in streetview imagery and machine learning has allowed for further understanding into the visual aspect of gentrification, by means of large scale analyses of images of facades. However, there has not been much focus on the storefront signage (shop signs and other textual elements) with regards to this topic; while previous studies have pointed out that certain aesthetics of signage (e.g. fonts, colors, semantics) are associated to gentrification. This research aim to fill this gap, by using computer vision techniques on the crowd-sourced StreetSwipe dataset of streetview images in Amsterdam, to understand which attributes of storefront signage account for perceived gentrification.
\end{abstract}

\keywords{gentrification, computer vision, streetview imagery, storefront signage, scene-text analysis}

%%%%%%%%%%%%%%%%%%%%%%%%%%%%%%%%%%%%%%%%%%%%%%%%%%%%%%%%%%%%%%%%%%%%%%%%%%%%%%%%
% CONTENT
%%%%%%%%%%%%%%%%%%%%%%%%%%%%%%%%%%%%%%%%%%%%%%%%%%%%%%%%%%%%%%%%%%%%%%%%%%%%%%%%
\begin{document}

\pagestyle{plain}
\setcounter{page}{1}
\fixemptypage

\maketitle

% Sections; Try to stick to this setup
\textbf{Github:} https://github.com/atran13/MSc-Thesis-Gentrification-and-storefront-signage-in-Amsterdam 

\section{Introduction}
\label{sec:introduction}
% Mention scientific context/field, problem statement, research gap and candidate (sub) research question(s). 

Within urban studies, gentrification is a phenomenon widely discussed. First coined by British sociologist Ruth Glass in 1964 in her work about the inner city of London \cite{Glass1964}, the term refers to a neighborhood changing as a result of wealthier residents moving in, gradually displacing existing residents as local housing and service prices increase, and culture homogenized or replaced. Gentrification is a multi-dimensional, multi-step socio-spatial process. It involves functional, symbolic, architectural and social aspects \cite{döring_ulbricht_2018} - all of which are shown in economic and demographic shifts, as well as changes in the aesthetics of the built environment.

For this reason, in researching gentrification, next to looking into economic and demographic indicators, visual indicators such as improvements in the neighborhood's physical appearance and changes in design style are very often analyzed \cite{huang2022, ravuri2022, naik2017, ilic2019}. Comparisons have been drawn between gentrified and non-gentrified facades in terms of old versus new features, openness of the properties (e.g. boarded up windows, fences), greenery, colors,...; but not as much attention has been paid to textual features, namely storefront signage. Most research that has been done so far in this regards are in the context of the US, in which clear distinctions were made between gentrified and non-gentrified storefront signage. The current study, therefore, aim to examine attributes of signage associated with gentrification, utilizing the Amsterdam streetview image dataset from the StreetSwipe project, thus adding to the understanding of gentrification in Amsterdam.

Using crowd-sourcing, StreetSwipe \cite{streetswipe} lets people decide whether each facade in Amsterdam appears gentrified. By using this data, this study draws conclusions based on subjective and common perception of a diverse group of people - arguably a necessity when it comes to understanding a nuanced and multi-faceted phenomenon such as gentrification. The data is analyzed using machine learning techniques, namely Convolution Neural Networks to recognize and extract storefront signage texts and colors, and word embedding to analyze the semantics of texts. In doing so, this study sets out to systematically uncover characteristics of storefront signage that has been classified as gentrified or non-gentrified - in other words, to see what matters to people's perception when judging a facade with regards to gentrification, in terms of text font, color, and semantic. Furthermore, the study aim to apply the learnt characteristics on a bigger set of streetview data of Amsterdam in order to identify potential areas of gentrification in the city - areas that share the same aesthetics as previously labelled data. The research question of this study is stated as follow: 

\noindent\textit{To which extent can scene-text machine learning methods applied on streetview images of storefronts help identify the attributes of signage texts associated to perceived gentrification, and identify potential areas of gentrification in Amsterdam?}

To answer this research question, this study aims to clarify the following sub-questions: 

\begin{enumerate}
    \item To which extent can font types be extracted accurately from gentrified and non-gentrified storefront images?
    \item To which extent can text colors be extracted accurately from gentrified and non-gentrified storefront images?
    \item To which extent can text semantic be extracted accurately from gentrified and non-gentrified storefront images?
    \item Which characteristics of font, color, and semantic are related to gentrification?
    \item How do the features extracted from gentrified/non-gentrified signage distribute in a larger dataset of Amsterdam streetview images?
\end{enumerate}

\section{Related Work}
\label{sec:related_work}

This section first outlines the findings of current ethnographic and urban studies on signage and gentrification, followed by a brief exploration into scene-text attribute learning. Next, the state-of-the-art of scene-text detection are given. Finally, image classification is discussed and the computer vision model to be used in the study is presented.

\subsection{Signage and gentrification}

As aforementioned, the current body of research about storefront signage aesthetics in relation to gentrification exists largely in the context of the US. While some studies took a holistic approach and considered multiple elements, namely font type (or typeface), colors, text density, language, and meaning, others analyzed elements individually.

Findings from Brooklyn, New York\cite{trinch_signsays_2017, snajdr_oldschool_2018, snajdr_preserve_2022} show that non-gentrified signage - or as the authors call them: \textit{old-school} signage - typically is text-dense and has larger typeface; names that refer to the location, the owner's name, the type of business, products or services; languages other than English; complementary symbols or images; reference to religion, ethnicity, country of origin, and race. On the contrary, gentrified signage - or \textit{distinction-making} signage - has shorter texts, written in smaller font sizes, lower case letters; more cryptic or ambiguous names, sometimes polysemic and with word-plays; languages other than English that shows sophistication and worldliness. In parallel, a study from Cincinnati, Ohio \cite{rahman_signage_2020} found that signage forms and characters became more homogeneous as neighborhoods became gentrified.

Next to this, other studies have also analyzed storefront signage but with their focus on individual attributes. A popular topic within gentrification in this regard is the linguistic landscape of a city or neighborhood. Examples can be found in Seoul, South Korea \cite{hong_linguistic_2020}: in the neighborhood where the majority of the Chinese population in Seoul reside, Korean signage has been gradually replaced by Chinese signage, mirroring both the demographic composition and the social and economic standing of these groups. Similarly, more English signage has appeared in a commercial neighborhood in Phnom Penh, Cambodia, seemingly displacing French as a second language - a trend observed alongside globalization, gentrification, a generational change in attitudes, and education policies \cite{kasanga_map_2012}. Besides languages in signage, typeface has also been found to be highly correlated with average household income of the corresponding neighborhoods in London \cite{ma_typef_2019}.

All in all, findings show clear differences between the characteristics of gentrified and non-gentrified signage. It was therefore hypothesized that the same pattern can be found in Amsterdam's signage, and thus a computer vision model can be used to distinguish and detect gentrification based on signage.

\subsubsection{Scene-text attribute learning}
Taking inspiration from the insights outlined above, where signage was analyzed in terms of typeface, color, and textual meaning (among other elements), an exploration in this direction was done for the current study. The initial goal was to learn these specific attributes and quantify, on a large scale, what characteristics differentiate gentrified and non-gentrified signage in Amsterdam. The pipeline would involve detecting the text instances from the original images (text detection), classifying the font type, inferring the dominant colors, transcribing the text (text recognition), and learn semantics via word embeddings. However, given the state of the dataset at hand - namely, the lack of annotations other than the gentrification label - as well as resources available, this research direction was ultimately deemed unsuitable. The state-of-the-art approaches are described below.

\begin{enumerate}
    \item End-to-end: These are models that can take an image and output multiple elements of interest. Examples are NeurTEx \cite{aggarwal_neurtex_2022} (outputs text bounding box, transcript, and typeface), Fontnet \cite{s_fontnet_2021} (outputs color and typeface), and TaCo \cite{nie_taco_2022} (outputs color and typeface). However, these models are designed to be applied in the context of graphic design and on documents. They are not reliable when it comes to scene-text because of the added noise and distortions of natural images.
    \item Step-by-step: This approach uses individual models to learn each element:
    \begin{itemize}
        \item Font: State-of-the-art models include DeepFont \cite{wang_deepfont_2015}, HENet \cite{chen_henet_2021}, and benchmark datasets include AdobeVFR \cite{wang_deepfont_2015}, and VFR-2420 \cite{chen_large-scale_2014}. These data are largely synthetic, and accordingly the models perform well on synthetic data, but are not as robust when it comes to natural images. Furthermore, with neither pre-trained models nor ground truth available on StreetSwipe images, training and evaluation proves to be challenging and unreliable.
        \item Text transcripts and semantics: State-of-the-art scene-text recognition models include MORAN \cite{luo_multi-object_2019}, CRNN \cite{shi_end2end_2015}, and PARSeq \cite{bautista_scene_2022}. For learning semantics, word embedding models such as FastText \cite{bojanowski_enriching_2017} or Word2Vec \cite{mikolov_efficient_2013} could be applied on the transcribed text. However, as was found from text detection (discussed in a later section), the text instances from the data were either not meaningful words (a nature of signage text), or fragmented upon detection (words are broken up). Learning semantics from this data would not give meaningful and consistent results.
        \item Color: Text colors as a standalone feature could be learnt via creating a color histogram \cite{srivastava_review_2015} per image. But instead, by using a convolution neural network (discussed in a later section), colors can be learned as well along with many other features.
    \end{itemize}
\end{enumerate}

To account for the lack of ground truth labels, it was also considered that a synthetic scene-text dataset is created to be used as training and testing data, before inference is done on StreetSwipe. Gupta et al. \cite{gupta_synt_2016} offers a method for this that takes into account image segmentation and depth in order to generate text, with annotations on bounding box, typeface, color, and text transcript as needed. An example usage in a closely related study was done by Ma et al. \cite{ma_typef_2019}, in which the authors generated training data to recognize typeface on signage. Considering time constraint, however, this approach was non-viable. It required either mining street view images that do not have text to be synthesized on, or using pre-generated images from Gupta et al., which largely included many types of background other than street view of facades. Either method proved impractical - the former is time-consuming, even when discounting for model training and tuning; and the latter is a domain mismatch compared to StreetSwipe.

All in all, a machine learning approach for replicating existing studies was not found. A more appropriate research direction was devised that involved learning a more general visual representation of the signage with a convolutional neural network. Signs would firstly be extracted using a pre-trained text-detection model, and a classification model would be trained and tested on these signage.

\subsection{Scene-text detection}
Scene-text detection involves identifying and localizing text in natural images - a task fundamental to the current study. The challenge in this task stems from extracting text from complex images, with the text surrounded by other objects, varying in sizes, perspectives, orientations, sometimes curved, obstructed, or blurry.

Benchmark datasets for this task include ICDAR 2013 \cite{icdar13} and 2015 \cite{icdar15}, and TotalText \cite{totaltext}. The most widely implemented models are EAST \cite{zhou_east_2017} and CRAFT \cite{baek_character_2019}. CRAFT is the best performing model on ICDAR 2013, and surpasses EAST on ICDAR 2015 and TotalText. CRAFT is more accurate on curved, long, and non-horizontal text. By calculating character region scores (localizing characters) and affinity scores between characters (grouping characters into sequences), the model returns word-level bounding boxes. The model is also multilingual - a necessary feature considering signage in the dataset at hand are (at least) in Dutch, English, Chinese and Korean. CRAFT is thus the model used in this study for extracting signage.

The pre-trained CRAFT model was implemented via the EasyOCR Python package \cite{noauthor_jaided_nodate}, which offered an ease of implementation and also allowed for processing multiple languages simultaneously.

\subsection{Image classification}
The Residual Network (ResNet) \cite{resnet}, with the use of skip connections in its architecture, has enabled deeper networks to learn more efficiently, without the problem of vanishing or exploding gradients. For the task of learning and classifying signage, fine-tuned ResNet18 and ResNet50 were chosen as candidate models, initialized with pre-trained weights from ImageNet. After achieving satisfactory performance, the best model's predictions were further analyzed. 

Firstly, the correctly classified signage within the top 20 highest probabilities from each class were inspected - this showed the most typical and distinguishing characteristics of gentrified and non-gentrified signage. In comparison with existing studies (as outlined in section 2.1), conclusions were drawn on the extent to which patterns found elsewhere were similar to our case of Amsterdam.

Secondly, the incorrectly classified signage were inspected. These can be considered the more fuzzy cases - something that previous studies did not report on.

Finally, the model was tested on an external dataset of other neighborhoods in the city. Given literature on gentrification in Amsterdam, gentrified and non-gentrified neighborhoods were selected, and street view images from these areas were retrieved from a dataset made available by the Civic AI Lab. Gentrified areas include Jordaan \cite{verlaan_hippies_2022}, Oud-West, De Baarjes \cite{rettberg_when_2019}; and non-gentrified areas include Zuidoost and Nieuw-West \cite{pinkster_stickiness_2020}. Images taken from these neighborhoods are labeled accordingly. Thus, when testing the model (trained on visual perception) on this data (labeled per census data), we could tell to which extent the perception is applicable on a neighborhood level. In other words, given a gentrified neighborhood, to which extent are signage in that neighborhood seen as gentrified? Further, the model's predictions on this data were also visualized as per the highest 20 probabilities per class, irrespective of ground truth labels. This showed what the model considered as gentrified and non-gentrified, and were this to be in-line with characteristics found in StreetSwipe, this would support the model's generalizability in detecting gentrification.


\section{Methodology}
\label{sec:methodology}

\subsection{Data}
\subsubsection{StreetSwipe}
The dataset with gentrified and non-gentrified labels is retrieved from the Streetswipe project. Using crowdsourcing, the project let people decide whether each Amsterdam facade is gentrified, by voting "Yes" or "No" on the streetview images of these facades. The official \textit{Gentrified} and \textit{Non-gentrified} labels are based on what the majority of people voted for, for each facade. Additionally, if subsequent voters decide against the majority (e.g. voting \textit{Gentrified} for a non-gentrified-labelled facade), they are also prompted to provide a textual reasoning for their decision. These mismatch responses are also available, however this is out of scope of the study. 

Since there are two versions of StreetSwipe, the data retrieved exists in two sets, consisting of 1912 higher resolution images from the older version and 530 lower resolution images from the new one, each with its label as well as proportions of "Yes" and "No" votes. The StreetSwipe dataset thus have 2,442 images in total, from which the characteristics of gentrified/non-gentrified storefront signage will be extracted.

\subsubsection{Synthetic scene-text data}
In order to recognize signage fonts, it is planned that the dataset and pre-trained model from Ma et al. [2019] \cite{ma2019} are used. Their study focused on detecting and recognizing typeface in streetview imagery of London neighborhoods to ultimately analyze the relationship between typeface and neighborhood income. In order to achieve this, they synthesized a dataset using the method of Gupta et al. [2016] \cite{gupta2016} for text localization in natural images, taking into account the depth and segmentation of the images. The resulting dataset consists of 91,398 streetview images with text added onto appropriate areas, with 11 typefaces of the text (sans serif, decorative, script, serif, etc.) as ground truth labels. On this data, they trained a ResNet-18 for typeface recognition. 

Since manually labelling typefaces from the StreetSwipe images would be time consuming and require professional knowledge, having the pre-trained model for typeface recognition and the synthesized typeface data would be more suitable for the scope of the study. The next step to gather this data is to contact the authors of this study, since the data is not publicly available.

\subsubsection{Amsterdam panoramic data}
The last dataset aimed to be included in this study is a panoramic streetview data of Amsterdam. It is intended that the complete models will be applied on this data to learn the city-wide distribution of gentrified and non-gentrified storefront signage, based on the attributes associated with gentrification; and thus conclusions can be drawn about potential areas of gentrification in the city. Currently, the shape of the dataset is not yet known.

\subsection{Approach}
Models will first be trained and tested on the synthetic scene-text data, to learn typeface and color of the text. This dataset will be used for training and testing because it has ground truth label of the typeface and text color, and the most reliable size. After achieving satisfactory performance metrics on this data, the trained models will be applied to the StreetSwipe dataset to recognize typeface and color of signage, in association with the gentrified/non-gentrified labels. On this data, an unsupervised semantic analysis will also be done on the extracted text. And finally, having known which characteristics are indicative of gentrification, the models will be applied to the panoramic Amsterdam dataset to identify the areas in which these characteristics are present.

The following steps will be taken at each stage of the research (with input images from each dataset):
\begin{enumerate}
    \item \textbf{Scene-text detection}: Detecting text in an image, by creating bounding boxes using CRAFT (Character-Region Awareness For Text detection). Compared to another state-of-the-art text detection model - EAST (Efficient and accurate scene text detector) - CRAFT is more accurate and is multi-lingual.
    \item \textbf{Color detection} Create color histograms for the distribution of colors present in the text box.
    \item \textbf{Typeface recognition} Recognize the font of the text using Ma et al. [2019] pre-trained model.
    \item \textbf{Scene-text recognition} Transcribe the image-text into text strings readable by models, using MORAN (Multi-object rectified attention network).
    \item \textbf{Semantic analysis} Transform the extracted texts into a word vector representation using FastText, to recognize languages and semantics such as cryptic or literal store names, inclusion of ethnicity, religions, etc.
\end{enumerate}


% ALTERNATIVELY
% \subsubsection{ST-Sem}
% Trained to classify points of interest of facades
% Scene recognition using ResNet152-places365 pre-trained model
% + Scene-text detection using TextBoxes++
% + Scene-text recognition using MORAN
% + Semantic matching using FastText (a vector space model)

\subsection{Evaluation}
Evaluation is done with the goal to see how accurate the models for extracting fonts, colors, and semantic are. For font and color, this is aimed to be done with the Ma et al.'s synthetic data, with an 80-20 split for the train and test set, respectively. It is appropriate as it has ground truth labels for font and color of text, whereby precision and recall will be calculated. As for the word embedding model, since this is unsupervised, evaluation will be done by examining the word vector's nearest neighbors to see the semantic information the vector captured.
\section{Risk Assessment}
\label{sec:risk_assessment}

A risk that this project could run into is not getting synthetic data and/or the pre-trained model from Ma et al. [2019]. This could be due to the authors being unable or unwilling to share the data. In this case, other datasets for scene-text recognition are publicly available, such as ICDAR 2013 and SVT (Street view text), albeit in smaller sizes, and extra work would need to be done to annotate the data with relevant labels.
\section{Project Plan}
\label{sec:project_plan}

\begin{table}[H]
    \centering
    \begin{tabular}{p{0.8cm}p{0.6cm}p{1.6cm}p{4.3cm}} \toprule \textbf{Month} & \textbf{Week} & \textbf{Date} & \textbf{Achievement} \\ \midrule
        % February & 4 & 20/02 - 24/02 & Finish literature research and finalize thesis design. Contact for gathering dataset. \\ 
        March & 1 & 27/02 - 03/03 & Gather remaining data. Complete writing introduction. \\
            & 2-3 & 06/03 - 17/03 & Have data structured in a ready-to-use format, complete exploratory data analysis for \textbf{milestone 2}. \\
            & 4-5 & 20/03 - 31/03 & Train text detection and color detection models. Complete writing related work. \\
        April & 1 & 03/04 - 07/04 & Train text recognition model. \\
            & 2 & 10/04 - 14/04 & Evaluation of typeface (pre-trained) and color recognition models on test data. Complete writing methodology for \textbf{milestone 3.} \\
            & 3-4 & 17/04 - 28/04 & Implement typeface, color and word-embedding models on StreetSwipe data, get results per category (gentrified and non-gentrified). \\
        May & 1-2 & 01/05 - 12/05 & Implement models on Amsterdam panoramic data, compare distributions of attributes, draw conclusions. \\
            & 3 & 15/05 - 19/05 & Complete writing results for \textbf{milestone 4.} \\
            & 4-5 & 22/05 - 02/06 & Write discussion, conclusion and future work. \\
        June & 1 & 05/06 - 09/06 & Incorporate feedback on previous parts. \\
            & 2 & 12/06 - 16/06 & Complete thesis draft for \textbf{milestone 5.} \\
            & 3-4 & 19/06 - 30/06 & Revise and complete report, hand in final version. \\
    \end{tabular}
    \caption{Project plan per week}
    \label{tab:my_label}
\end{table}


\bibliographystyle{ACM-Reference-Format}
\bibliography{bibliographies/references}

\end{document}

%%%%%%%%%%%%%%%%%%%%%%%%%%%%%%%%%%%%%%%%%%%%%%%%%%%%%%%%%%%%%%%%%%%%%%%%%%%%%%%%
%%%%%%%%%%%%%%%%%%%%%%%%%%%%%%%%%%%%%%%%%%%%%%%%%%%%%%%%%%%%%%%%%%%%%%%%%%%%%%%%