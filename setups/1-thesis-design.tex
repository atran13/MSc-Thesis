%%%%%%%%%%%%%%%%%%%%%%%%%%%%%%%%%%%%%%%%%%%%%%%%%%%%%%%%%%%%%%%%%%%%%%%%%%%%%%%%
% setup.tex
% Main tex file for setup and content collocation
% For the use of University of Amsterdam
% Information Systems and Data Science students
% Adapted by Riccardo Fiorista (riccardo.fiorista@proton.me)
%%%%%%%%%%%%%%%%%%%%%%%%%%%%%%%%%%%%%%%%%%%%%%%%%%%%%%%%%%%%%%%%%%%%%%%%%%%%%%%%

% Options:
%% `nolinenumbering` - If you want to remove line numbering on submission
%% `draftmargins` - If you would like to give your reviewer more space for comments
\documentclass[nolinenumbering]{thesisdesign}

%%%%%%%%%%%%%%%%%%%%%%%%%%%%%%%%%%%%%%%%%%%%%%%%%%%%%%%%%%%%%%%%%%%%%%%%%%%%%%%%
% DOCUMENT METADATA
%%%%%%%%%%%%%%%%%%%%%%%%%%%%%%%%%%%%%%%%%%%%%%%%%%%%%%%%%%%%%%%%%%%%%%%%%%%%%%%%

% Thesis related entries
\title{Gentrification in Storefront Signage in Amsterdam: A Machine Learning Approach with Streetview Imagery}
\subtitle{Submitted on: \textbf{19-02-2023}}

% Author data
\authorname{Author: Anh Tran}
\authoremail{anh.tran1@student.uva.nl}
% ACM-sepecific entries
\affiliation{
    \institution{\thesisinstitution}
    \city{\thesiscity}
    \country{\thesiscountry}
}

% Supervisor data
\authorname{Supervisor: Tim Alpherts}
\authoremail{t.o.l.alpherts@uva.nl}
% ACM-sepecific entries
\affiliation{
    \institution{\thesisinstitution}
    \city{\thesiscity}
    \country{\thesiscountry}
}

\begin{abstract}
    Gentrification refers to the process of a neighborhood changing as a result of wealthier residents moving in, bringing investments and improvements, but displacing the existing residents due to rising prices and changing cultures. More recent developments in streetview imagery and machine learning has allowed for further understanding into the visual aspect of gentrification, by means of large scale analyses of images of facades. However, there has not been much focus on the storefront signage (shop signs and other textual elements) with regards to this topic; while previous studies have pointed out that certain aesthetics of signage (e.g. fonts, colors, semantics) are associated to gentrification. This research aim to fill this gap, by using computer vision techniques on the crowd-sourced StreetSwipe dataset of streetview images in Amsterdam, to understand which attributes of storefront signage account for perceived gentrification.
\end{abstract}

\keywords{gentrification, computer vision, streetview imagery, storefront signage, scene-text analysis}

%%%%%%%%%%%%%%%%%%%%%%%%%%%%%%%%%%%%%%%%%%%%%%%%%%%%%%%%%%%%%%%%%%%%%%%%%%%%%%%%
% CONTENT
%%%%%%%%%%%%%%%%%%%%%%%%%%%%%%%%%%%%%%%%%%%%%%%%%%%%%%%%%%%%%%%%%%%%%%%%%%%%%%%%
\begin{document}

\pagestyle{plain}
\setcounter{page}{1}
\fixemptypage

\maketitle

% Sections; Try to stick to this setup
\textbf{Github:} https://github.com/atran13/MSc-Thesis-Gentrification-and-storefront-signage-in-Amsterdam 

\section{Introduction}
\label{sec:introduction}
% Mention scientific context/field, problem statement, research gap and candidate (sub) research question(s). 

Within urban studies, gentrification is a phenomenon widely discussed. First coined by British sociologist Ruth Glass in 1964 in her work about the inner city of London \cite{Glass1964}, the term refers to a neighborhood changing as a result of wealthier residents moving in, gradually displacing existing residents as local housing and service prices increase, and culture homogenized or replaced. Gentrification is a multi-dimensional, multi-step socio-spatial process. It involves functional, symbolic, architectural and social aspects \cite{döring_ulbricht_2018} - all of which are shown in economic and demographic shifts, as well as changes in the aesthetics of the built environment.

For this reason, in researching gentrification, next to looking into economic and demographic indicators, visual indicators such as improvements in the neighborhood's physical appearance and changes in design style are very often analyzed \cite{huang2022, ravuri2022, naik2017, ilic2019}. Comparisons have been drawn between gentrified and non-gentrified facades in terms of old versus new features, openness of the properties (e.g. boarded up windows, fences), greenery, colors,...; but not as much attention has been paid to textual features, namely storefront signage. Most research that has been done so far in this regards are in the context of the US, in which clear distinctions were made between gentrified and non-gentrified storefront signage. The current study, therefore, aim to examine attributes of signage associated with gentrification, utilizing the Amsterdam streetview image dataset from the StreetSwipe project, thus adding to the understanding of gentrification in Amsterdam.

Using crowd-sourcing, StreetSwipe \cite{streetswipe} lets people decide whether each facade in Amsterdam appears gentrified. By using this data, this study draws conclusions based on subjective and common perception of a diverse group of people - arguably a necessity when it comes to understanding a nuanced and multi-faceted phenomenon such as gentrification. The data is analyzed using machine learning techniques, namely Convolution Neural Networks to recognize and extract storefront signage texts and colors, and word embedding to analyze the semantics of texts. In doing so, this study sets out to systematically uncover characteristics of storefront signage that has been classified as gentrified or non-gentrified - in other words, to see what matters to people's perception when judging a facade with regards to gentrification, in terms of text font, color, and semantic. Furthermore, the study aim to apply the learnt characteristics on a bigger set of streetview data of Amsterdam in order to identify potential areas of gentrification in the city - areas that share the same aesthetics as previously labelled data. The research question of this study is stated as follow: 

\noindent\textit{To which extent can scene-text machine learning methods applied on streetview images of storefronts help identify the attributes of signage texts associated to perceived gentrification, and identify potential areas of gentrification in Amsterdam?}

To answer this research question, this study aims to clarify the following sub-questions: 

\begin{enumerate}
    \item To which extent can font types be extracted accurately from gentrified and non-gentrified storefront images?
    \item To which extent can text colors be extracted accurately from gentrified and non-gentrified storefront images?
    \item To which extent can text semantic be extracted accurately from gentrified and non-gentrified storefront images?
    \item Which characteristics of font, color, and semantic are related to gentrification?
    \item How do the features extracted from gentrified/non-gentrified signage distribute in a larger dataset of Amsterdam streetview images?
\end{enumerate}

\section{Related Work}
\label{sec:related_work}

% Current research is scarce when is comes to understanding the textual attributes of gentrification outside of the US, much less studies that leverage the potential of machine learning and streetview imagery. The following sections clarify this gap in research.

\subsection{Gentrification}

% In examining which factors lead to gentrification based on the existing body of research, Rigolon and Németh \cite{rigolon2019} devised a socioecological model in which people, place, and policy are identified as variables that together shape neighborhood change. The \textit{people} layer includes the neighborhood's demographic and socioeconomic characteristics, such as race, income, and the strength of the community culture. For instance, a high proportion of people of color can limit the investment the area receives from developers, as well as how fast it is "discovered" by gentrifiers, due to stigma and perceptions regarding safety. The \textit{place} layer comprises the neighborhood characteristics, such as its proximity from downtown, presence of old but appealing buildings, presence of green space, accessibility by public transit, schools, offices - all of which gives the neighborhood more redevelopment potential and higher likelihood of gentrification. Lastly, the \textit{policy} layer refers to actions by the authorities and other local interventions to prevent gentrification and displacement. While efforts such as providing and protecting housing mitigate gentrification, investing in public infrastructure in low-income neighborhoods leads to the opposite effect.

Döring and Ulbricht \cite{döring_ulbricht_2018} define gentrification as having four dimensions: functional, architectural, social, and symbolic. 
\begin{itemize}
    \item The functional aspect refers to the establishment of services, businesses, and cultural institutions, often with a recognized name and better quality than those existing in the neighborhood. An example is a supermarket chain, as opposed to an ethnic grocery store.
    \item The architectural aspect concerns upgrades made to the built environment, and the changing aesthetics that comes along with them. Such upgrades is done to residential as well as social infrastructures, such as public buildings, sidewalks, parks, etc. 
    \item The symbolic aspect refers to the new image of the gentrifying neighborhood created by the new residents as well as other stakeholders, such as investors and the media, and communicated to the wider public.
    \item The social aspect concerns the displacement of existing residents and replacement by those with more socioeconomic capital. Economically marginalized by rising costs, and socially marginalized by a changing neighborhood culture, long-established groups of residents find themselves no longer belonging, and the demographic of the neighborhood thus gradually transforms.
\end{itemize}

As has been noted by Feiereisen and Sassin \cite{feiereisen2021}, the functional, architectural, and symbolic aspects can be seen as constituting the visual indicators of gentrification. While these three aspects take shape in multiple elements and characteristics of the built environment, textual signage is the one element where they converge. A storefront signage would ideally communicate the functionality of the place, while having a certain style of design that reflects and fits in with the overall architectural design of the place, and altogether convey a certain symbolic image, one that gives passerby an impression of what the place stands for, its atmosphere and aesthetics. 

\begin{displayquote}
    "Shop signs are public texts that communicate what stores sell, who is perceived to be on the street and what their commercial desires are thought to be. [...] Similar to spoken utterances and all written texts, signs are designed for particular audiences [...]. Well-crafted stories are place-making tools inasmuch as they maintain and reproduce prevailing cultural standards and values" \cite{trinch_signsays_2017}. 
\end{displayquote}

Analyzing storefront signage can thus reveal a lot about the visual as well as semantic pattern of perceived gentrification, and this is where the interest of the current research lies.

% \subsubsection{Gentrification in Amsterdam}

\subsection{Storefront signage and gentrification}

As aforementioned, the current body of research about storefront signage aesthetics in relation to gentrification exists largely in the context of the US. To clarify, these are studies that took a holistic approach and considered multiple elements, namely font type (or typeface), colors, text density, language, and meaning. For a sign to convey functional, architectural, and symbolic features of gentrification, it is not up to one single attribute of its appearance, but rather these multiple elements coming into play. Functions would not be expressed effectively through text font and color, but more through semantic meaning, while the opposite can be said about architectural design. And as neighborhood residents see a store sign, they do not just notice one thing, but rather take in its color, font, and meaning as a whole, and in turn form their perception of the place. Therefore, it is deemed necessary that the current analysis take into account multiple attributes of storefront signage, in order to capture as completely as possible the visual cues of gentrification, as experienced by any observer.

Findings from Brooklyn, New York\cite{trinch_signsays_2017, snajdr_oldschool_2018, snajdr_preserve_2022} show that non-gentrified signage - or as the authors call them: \textit{old-school} signage - typically is text-dense and has larger typeface; names that refer to the location, the owner's name, the type of business, products or services; languages other than English; complementary symbols or images; reference to religion, ethnicity, country of origin, and race. On the contrary, gentrified signage - or \textit{distinction-making} signage - has shorter texts, written in smaller font sizes, lower case letters; more cryptic or ambiguous names, sometimes polysemic and with word-plays; languages other than English that shows sophistication and worldliness. In parallel, a study from Cincinnati, Ohio \cite{rahman2020} also found that signage form and character became more homogeneous as neighborhoods became gentrified.

Next to this, other studies have also analyzed storefront signage but with their focus on one individual attribute. A popular topic within gentrification in this regard is the linguistic landscape of a city or neighborhood, for which storefront signage serves as the most abundant proof. Examples can be found in Seoul, South Korea \cite{hong2020}: in a small ethnic neighborhood called Garibong-dong, where the majority of the Chinese population in Seoul reside, Korean signage has been gradually replaced by Chinese signage throughout the years, mirroring both the demographic composition and the social and economic standing of these groups. Similarly, more English signage has appeared in a commercial neighborhood in Phnom Penh, Cambodia, seemingly displacing French as a second language - a trend observed alongside globalization, gentrification, a generational change in attitudes, and education policies \cite{kasanga2012}. Besides languages in signage, typeface has also been found to be highly correlated with average household income of the corresponding neighborhoods in London \cite{ma2019}.

All in all, it is worthwhile to expand knowledge on the topic of storefont signage and gentrification to other regions, not only because there are clear differences between the aesthetics of gentrified and non-gentrified signage, but also because these patterns of change could be different among countries and cultures. The final part of this section discusses the resources at hand for this task, namely streetview imagery (in conjunction with the Streetswipe dataset of Amsterdam), and machine learning in studying gentrification.

\subsection{Machine learning and Streetview imagery}

There exist multiple gentrification research using machine learning techniques: predictive models have been developed for gentrification in Sydney \cite{thackway2021} and London \cite{reades2019}, based on historic socio-economic transitions. Coupled with the increasing availability of street-level image data - or streetview imagery, this has enabled researchers of urban studies to study visual attributes and patterns of change overtime in cities on a significantly larger scale. Besides the previously mentioned studies on linguistic landscape in Seoul \cite{hong2020} and typeface and neighborhood income in London \cite{ma2019}, which utilized streetview imagery and machine learning, other work has been done to visually measure gentrification via documenting changes \cite{ravuri2022}, detect \cite{huang2022} and deep-map gentrification to reveal areas unknowingly becoming gentrified \cite{ilic2019}. Systematic literature reviews on streetview imagery and computer vision for urban analytics \cite{biljecki_2021, zhanga2023} have noted the applicability and increasing importance of the approach in generating insights and decision-making, while pointing future research to further analyzing written languages in images, as well as between-place inference (applying a machine learning model trained with one area to another area). By studying storefront signage of gentrified/non-gentrified labelled data and applying the learnt attributes to a larger set of Amsterdam data, this study adds to this gap in research. Additionally, computer vision research done with storefront signage so far has only been to classify points of interest \cite{noorian2020, bakaev2019}, whereas the current study utilized these resources to understand the visual attributes of signage by analyzing typeface, color, and semantic.

% - SVI and crowdsourced data for urban visual intelligence / place perception, e.g Place Pulse (Boston, New York, Linz, and Salzburg), Place Pulse 2.0 (a more globally diversed version)
% --> More diversed opinions/view points; also more consistent and reliable

\section{Methodology}
\label{sec:methodology}

\subsection{Data}
\subsubsection{StreetSwipe}
The dataset with gentrified and non-gentrified labels is retrieved from the Streetswipe project. Using crowdsourcing, the project let people decide whether each Amsterdam facade is gentrified, by voting "Yes" or "No" on the streetview images of these facades. The official \textit{Gentrified} and \textit{Non-gentrified} labels are based on what the majority of people voted for, for each facade. Additionally, if subsequent voters decide against the majority (e.g. voting \textit{Gentrified} for a non-gentrified-labelled facade), they are also prompted to provide a textual reasoning for their decision. These mismatch responses are also available, however this is out of scope of the study. 

Since there are two versions of StreetSwipe, the data retrieved exists in two sets, consisting of 1912 higher resolution images from the older version and 530 lower resolution images from the new one, each with its label as well as proportions of "Yes" and "No" votes. The StreetSwipe dataset thus have 2,442 images in total, from which the characteristics of gentrified/non-gentrified storefront signage will be extracted.

\subsubsection{Synthetic scene-text data}
In order to recognize signage fonts, it is planned that the dataset and pre-trained model from Ma et al. [2019] \cite{ma2019} are used. Their study focused on detecting and recognizing typeface in streetview imagery of London neighborhoods to ultimately analyze the relationship between typeface and neighborhood income. In order to achieve this, they synthesized a dataset using the method of Gupta et al. [2016] \cite{gupta2016} for text localization in natural images, taking into account the depth and segmentation of the images. The resulting dataset consists of 91,398 streetview images with text added onto appropriate areas, with 11 typefaces of the text (sans serif, decorative, script, serif, etc.) as ground truth labels. On this data, they trained a ResNet-18 for typeface recognition. 

Since manually labelling typefaces from the StreetSwipe images would be time consuming and require professional knowledge, having the pre-trained model for typeface recognition and the synthesized typeface data would be more suitable for the scope of the study. The next step to gather this data is to contact the authors of this study, since the data is not publicly available.

\subsubsection{Amsterdam panoramic data}
The last dataset aimed to be included in this study is a panoramic streetview data of Amsterdam. It is intended that the complete models will be applied on this data to learn the city-wide distribution of gentrified and non-gentrified storefront signage, based on the attributes associated with gentrification; and thus conclusions can be drawn about potential areas of gentrification in the city. Currently, the shape of the dataset is not yet known.

\subsection{Approach}
Models will first be trained and tested on the synthetic scene-text data, to learn typeface and color of the text. This dataset will be used for training and testing because it has ground truth label of the typeface and text color, and the most reliable size. After achieving satisfactory performance metrics on this data, the trained models will be applied to the StreetSwipe dataset to recognize typeface and color of signage, in association with the gentrified/non-gentrified labels. On this data, an unsupervised semantic analysis will also be done on the extracted text. And finally, having known which characteristics are indicative of gentrification, the models will be applied to the panoramic Amsterdam dataset to identify the areas in which these characteristics are present.

The following steps will be taken at each stage of the research (with input images from each dataset):
\begin{enumerate}
    \item \textbf{Scene-text detection}: Detecting text in an image, by creating bounding boxes using CRAFT (Character-Region Awareness For Text detection). Compared to another state-of-the-art text detection model - EAST (Efficient and accurate scene text detector) - CRAFT is more accurate and is multi-lingual.
    \item \textbf{Color detection} Create color histograms for the distribution of colors present in the text box.
    \item \textbf{Typeface recognition} Recognize the font of the text using Ma et al. [2019] pre-trained model.
    \item \textbf{Scene-text recognition} Transcribe the image-text into text strings readable by models, using MORAN (Multi-object rectified attention network).
    \item \textbf{Semantic analysis} Transform the extracted texts into a word vector representation using FastText, to recognize languages and semantics such as cryptic or literal store names, inclusion of ethnicity, religions, etc.
\end{enumerate}


% ALTERNATIVELY
% \subsubsection{ST-Sem}
% Trained to classify points of interest of facades
% Scene recognition using ResNet152-places365 pre-trained model
% + Scene-text detection using TextBoxes++
% + Scene-text recognition using MORAN
% + Semantic matching using FastText (a vector space model)

\subsection{Evaluation}
Evaluation is done with the goal to see how accurate the models for extracting fonts, colors, and semantic are. For font and color, this is aimed to be done with the Ma et al.'s synthetic data, with an 80-20 split for the train and test set, respectively. It is appropriate as it has ground truth labels for font and color of text, whereby precision and recall will be calculated. As for the word embedding model, since this is unsupervised, evaluation will be done by examining the word vector's nearest neighbors to see the semantic information the vector captured.
\section{Risk Assessment}
\label{sec:risk_assessment}

A risk that this project could run into is not getting synthetic data and/or the pre-trained model from Ma et al. [2019]. This could be due to the authors being unable or unwilling to share the data. In this case, other datasets for scene-text recognition are publicly available, such as ICDAR 2013 and SVT (Street view text), albeit in smaller sizes, and extra work would need to be done to annotate the data with relevant labels.
\section{Project Plan}
\label{sec:project_plan}

\begin{table}[H]
    \centering
    \begin{tabular}{p{0.8cm}p{0.6cm}p{1.6cm}p{4.3cm}} \toprule \textbf{Month} & \textbf{Week} & \textbf{Date} & \textbf{Achievement} \\ \midrule
        % February & 4 & 20/02 - 24/02 & Finish literature research and finalize thesis design. Contact for gathering dataset. \\ 
        March & 1 & 27/02 - 03/03 & Gather remaining data. Complete writing introduction. \\
            & 2-3 & 06/03 - 17/03 & Have data structured in a ready-to-use format, complete exploratory data analysis for \textbf{milestone 2}. \\
            & 4-5 & 20/03 - 31/03 & Train text detection and color detection models. Complete writing related work. \\
        April & 1 & 03/04 - 07/04 & Train text recognition model. \\
            & 2 & 10/04 - 14/04 & Evaluation of typeface (pre-trained) and color recognition models on test data. Complete writing methodology for \textbf{milestone 3.} \\
            & 3-4 & 17/04 - 28/04 & Implement typeface, color and word-embedding models on StreetSwipe data, get results per category (gentrified and non-gentrified). \\
        May & 1-2 & 01/05 - 12/05 & Implement models on Amsterdam panoramic data, compare distributions of attributes, draw conclusions. \\
            & 3 & 15/05 - 19/05 & Complete writing results for \textbf{milestone 4.} \\
            & 4-5 & 22/05 - 02/06 & Write discussion, conclusion and future work. \\
        June & 1 & 05/06 - 09/06 & Incorporate feedback on previous parts. \\
            & 2 & 12/06 - 16/06 & Complete thesis draft for \textbf{milestone 5.} \\
            & 3-4 & 19/06 - 30/06 & Revise and complete report, hand in final version. \\
    \end{tabular}
    \caption{Project plan per week}
    \label{tab:my_label}
\end{table}


\bibliographystyle{ACM-Reference-Format}
\bibliography{bibliographies/references}

\end{document}

%%%%%%%%%%%%%%%%%%%%%%%%%%%%%%%%%%%%%%%%%%%%%%%%%%%%%%%%%%%%%%%%%%%%%%%%%%%%%%%%
%%%%%%%%%%%%%%%%%%%%%%%%%%%%%%%%%%%%%%%%%%%%%%%%%%%%%%%%%%%%%%%%%%%%%%%%%%%%%%%%